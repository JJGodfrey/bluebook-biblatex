\documentclass[12pt]{article}
\usepackage[style=oscola,terseinits=true]{biblatex}
\usepackage[style=british]{csquotes}
\usepackage[T1]{fontenc}
\usepackage[utf8]{inputenc}

\newcommand{\model}{%
  \par\strut\llap{*~}}

\setlength{\parindent}{0pt}
\setlength{\parskip}{1ex}

\addbibresource{oscola-examples.bib}

\title{OSCOLA Test Document}

\begin{document}

\maketitle

This document contains, one above the other, a handwritten canonically
correct example of various entry-types, as given in
\citetitle{oscola}\footcite{oscola} and the \texttt{oscola} version
for comparison.

\section{Books}

\model Andrew Burrows, \emph{Remedies for Torts and Breach of
  Contract} (3rd edn, OUP 2004) 317.

\cite[317]{burrows04}.

\model Justine Pila, \enquote{The Value of Authorship in the Digital
  Environment} in William H Dutton and Paul W Jeffreys (eds),
\emph{World Wide Research: Reshaping the Sciences and Humanities in
  the Century of Information} (MIT Press 2010).

\cite{pila10}.

\model John Cartwright, \enquote{The Fiction of the
  \enquote{Reasonable Man}} in AG Castermans and others (eds),
\emph{Ex Libris Hans Nieuwenhuis} (Kluwer 2009).

\cite{cartwright09}.

[NB The above uses \texttt{crossref} in the \texttt{.bib} file.]

\section{Cases}

\model \emph{Boulton v Jones} (1857) 2 H\&N 564, 157 ER 232.

\cite{boulton57}.

\model \emph{Henly v Mayor of Lyme} (1828) 5 Bing 91, 107; 130 ER 995,
1001.

\cite[107|1001]{henly28}.

\printbibliography

\end{document}

%%% Local Variables:
%%% coding: utf-8
%%% mode: LaTeX
%%% End:
