\documentclass[a5paper,fontsize=9pt,DIV=1]{scrartcl}

\usepackage[pass,a5paper]{geometry}

\usepackage[style=british]{csquotes}

\usepackage{hyperref}

\usepackage[style=oscola,
            indexing=cite,
            backend=biber,
            babel=hyphen]{biblatex}

\usepackage[splitindex,
            nonewpage]{imakeidx}

\setcounter{secnumdepth}{3}

\usepackage{tabularx}

\usepackage{fontspec}
\setmainfont[Ligatures = TeX,SmallCapsFont={Equity Caps B}]{Equity Text B}
\setsansfont[Ligatures = TeX]{Franklin Gothic Medium}

\usepackage{enumitem}
\setlist[description]{%
   font=\ttfamily\mdseries,
   leftmargin=3.8cm,
   labelwidth=3.6cm,
   labelsep=0.2cm}

\usepackage{multicol}

\usepackage{booktabs}

% Various commands and environments specific to this document
\newcommand{\oscola}{\textsc{bl-oscola}}
\newcommand{\biblatex}{\textsc{biblatex}}
\newcommand{\oscolashort}{\textsc{oscola}\nocite{oscola}}
\newcounter{egcounter}\setcounter{egcounter}{0}

\newenvironment{bibexample}[1][]{%
  \medskip\par\small\noindent\ignorespaces
  \marginpar{[\refstepcounter{egcounter}\arabic{egcounter}]\label{#1}}
  \begin{minipage}[t]{0.95\linewidth}}
 {\end{minipage}\par\medskip}

\newcommand{\egref}[1]{[\ref{#1}]}

\newcommand{\egcite}[1]{\texttt{\textbackslash cite#1}}
\newcommand\angledtext[1]{$\langle$\textit{#1}\/$\rangle$}

% BIBLIOGRAPHIC RESOURCES
\addbibresource{oscola-examples.bib}

% HYPHENATION
\hyphenation{am-end-ed}\hyphenation{nieu-wen-huis}
\hyphenation{sti-cht-ing}

% TYPOGRAPHY

\deffootnote{1em}{1em}{\thefootnotemark\ }
\frenchspacing

% INDEXES
\makeindex[name=ukcases, intoc=true,
           title={Table of UK Cases}]
\makeindex[name=eucasesa, intoc=true,
           title={Table of EU Cases (Alphabetical)}]
\makeindex[name=eucasesn, intoc=true,
           title={Table of EU Cases (Numerical)}]
\makeindex[name=intcases, intoc=true,
           title={Table of International Cases}]
\makeindex[name=ocases, intoc=true,
           title={Table of Cases from Other Jurisdictions}]
\makeindex[name=ukleg, intoc=true,
           title={Table of UK Legislation}]
\makeindex[name=euleg,intoc=true,
           title={Table EU Legislation and Treaties}]
\makeindex[name=treaties, intoc=true,
           title={Table of Treaties}]
\makeindex[name=pmats, intoc=true,
           title={Parliamentary Material and Draft Legislation}]
\makeindex[name=general,intoc=true,
           title={Index}]

\DeclareIndexAssociation{gbcases}{ukcases}
\DeclareIndexAssociation{encases}{ukcases}
\DeclareIndexAssociation{sccases}{ukcases}
\DeclareIndexAssociation{nicases}{ukcases}
\DeclareIndexAssociation{eucases}{eucasesa}
\DeclareIndexAssociation{eucasesnum}{eucasesn}
\DeclareIndexAssociation{echrcases}{intcases}
\DeclareIndexAssociation{echrcasescomm}{intcases}
\DeclareIndexAssociation{pilcases}{intcases}
\DeclareIndexAssociation{othercases}{ocases}
\DeclareIndexAssociation{gbprimleg}{ukleg}
\DeclareIndexAssociation{gbsecleg}{ukleg}
\DeclareIndexAssociation{enprimleg}{ukleg}
\DeclareIndexAssociation{ensecleg}{ukleg}
\DeclareIndexAssociation{scprimleg}{ukleg}
\DeclareIndexAssociation{scsecleg}{ukleg}
\DeclareIndexAssociation{cyprimleg}{ukleg}
\DeclareIndexAssociation{cysecleg}{ukleg}
\DeclareIndexAssociation{niprimleg}{ukleg}
\DeclareIndexAssociation{nisecleg}{ukleg}
\DeclareIndexAssociation{enroc}{ukleg}
\DeclareIndexAssociation{eutreaty}{euleg}
\DeclareIndexAssociation{euregs}{euleg}
\DeclareIndexAssociation{eudirs}{euleg}
\DeclareIndexAssociation{eudecs}{euleg}
\DeclareIndexAssociation{piltreaty}{treaties}
\DeclareIndexAssociation{gbdraftleg}{pmats}
\DeclareIndexAssociation{gbparltmat}{pmats}
\begin{document}

\title{The OSCOLA Biblatex Style}
\subtitle{Version 1.1}

\author{Paul Stanley\thanks{pstanley@essexcourt.net}}

\maketitle

\tableofcontents

\section{Scope}

\index[general]{scope of package|(}The \oscola\ style for \biblatex\
is intended to implement more or less the whole of the standard for
legal citations set out in the fourth edition of
\oscolashort,\footcite{oscola} in a way that as far as possible
respects the idea that bibliography databases should be style
independent.\index[general]{scope of package|)} However, since it
makes extensive use of non-standard entry-types (such as
\texttt{@jurisdiction}, \texttt{@legislation} and \texttt{@legal}), it
cannot be guaranteed that it will be completely consistent with other
legal styles.\footnote{For instance, I believe that the German
  \textsc{jura-diss} style uses the \texttt{author} field for the
  court that decides a case, rather than the \texttt{institution}
  field, as \oscola\ does.}

This document should be read alongside the \oscolashort\ standard
itself.\footcite{oscola} Wherever possible I have taken examples from
\oscolashort.

\subsection{Omissions\label{scope}}

\index[general]{scope of package!international materials} The most
recent edition of \oscolashort\ does not contain rules for formatting
international law materials (other than \textsc{eu} materials and
cases decided by the European Court of Human Rights). The 3rd edition
contained extensive rules.\footcite[25--37]{oscola3} The \oscolashort\
package offers \emph{basic} functionality for citing treaties and cases
consistently with the third edition; but the package does not attempt
to grapple with all the complexities, and might not be sufficient for
a work with \emph{extensive} citation of public international law
materials.

\textsc{Bl-oscola}\index[general]{scope of package!historical cases}
does not include citation forms for Yearbook or historical
cases\footcite[20--21]{oscola} (other than the \enquote{plain vanilla}
of the nominate reports and the English Reports). Nor does the package
contain rules for the citation of foreign legal\index[general]{scope
  of package!non-English materials} materials, other than US,
Canadian, Australian and New Zealand cases (which are cited
sufficiently often in English articles to justify including
them). There is no provision for citing legislation from those
countries.

To summarise:\index[general]{scope of package}
\begin{itemize}
\item The \oscola\ style provides full facilities for the citation of
  English, Scottish, Northern Irish, EU and European Convention cases,
  legislation and `official' materials.
\item The \oscola\ style provides full facilities for the citation
  of books, articles, encyclopaedias and looseleaf publications.
\item The \oscola\ style provides full coverage for the citation of
  materials such as online resources and private communications.
\item The \oscola\ style provides facilities for citing cases from
  the US, Australia, Canada and New Zealand, but not for citing
  legislation or `official' materials.
\item The \oscola\ style provides basic facilities for citing public
  international law treaties and case law, and at least many UN documents, but does not cover
  everything that the 3rd edition of \oscola\ specifies in relation to
  such materials.
\end{itemize}
I hope that the coverage is sufficient for most UK legal work; I believe
that only specialists in public international law or legal history are
likely to find it significantly deficient.

\subsection{Language}

\index[general]{language}Since \oscola\ is an English language
standard, the style has not been designed to support other
languages. Conversion to other languages would not necessarily be
unduly difficult; but it would involve making adjustments which are
likely to be troublesome except for very experienced users.

The style will, by default, set the \biblatex\ language option to
\texttt{english}.

\section{Some thanks, an apology, a warning, a promise}

\subsection{Thanks}

There are many people I should thank. \index[general]{Lehman,
  Philip}Philip Lehman and \index[general]{Kime, Philip}Philip Kime,
together with the current \biblatex\ maintainers Audrey Boruvka and
Joseph Wright -- without whom this would be impossible. Many members
of the community at
\TeX-StackExchange\footnote{\url{tex.stackexchange.com}} have helped
too. I've borrowed from more people than I can remember. Daniel
H\"ogger identified various important bugs, and suggested
improvements, particularly in relation to international law materials. Thanks to everyone.

\subsection{An apology}

Thanks \ldots\ and sorry. Sorry if you read the code, which is
sometimes muddled. Knowing what I know now, I would not start where I
started, or end where I ended. There's plenty of cleanup to be done,
but at least it works, and it seems better to release it, unpolished.

\subsection{A warning}

At least it works \ldots\ except when it doesn't. Of course there are
going to be problems: corner cases I haven't caught in testing, things
I've not thought about, decisions that turn out to be wrong. It's not
just not guaranteed not to be buggy, it's guaranteed to be
buggy.\index[general]{bugs!guaranteed}

\subsection{A promise}

It's guaranteed to be buggy \ldots\ but I promise to do what I
can. Not a legally binding promise, mind, but a promise binding in
honour that if something isn't working, and you need it working, I
will do my best to fix it as quickly as I can. Especially if you are
using this for serious work. Email
me. Really.\index[general]{bugs!reporting}

\section{For Absolute Beginners}

\index[general]{Biblatex!introduction
  to|(}\index[general]{Biber!introduction to|(}Few lawyers use \LaTeX,
so perhaps it would be reasonable to assume that those who do have a
fairly good idea of how it works. Nevertheless, this section gives a
\emph{very} short and simple introduction to using \biblatex, and
\textsc{biber} to create bibliographies. Infinitely more detail can be
found in the \biblatex\ documentation,\footcite{biblatex2} which is
essential reading.

The idea is this. You have two files: the file (or files) that
containing your document, and separate file(s) containing
bibliographical information about the works you may cite. Instead of
typing out a citation to the work in question, you use a simple
command to refer to it by a label you have selected and included in
your bibliography database. The system then takes care of all the
tedious details: formatting citations, keeping track of `ibid' or back
references to the first citation, giving full details in the
bibliography, making a table of cases and statutes, and so forth.

The work-flow is as follows:\index[general]{work-flow!in general} (1)
run \LaTeX, (2) run \textsc{biber}---the program that reads the files
produced by \LaTeX on its first run, and uses them to prepare
references in a digestible form for \LaTeX, (3) run \LaTeX\
again. Often you will also need (4) to run a program to construct
tables of cases and so forth, and (5) usually to run \LaTeX\ at least
once more to get everything in
order.\index[general]{Biblatex!introduction
  to|)}\index[general]{Biber!introduction to|)}

\subsection{What you need}

\index[general]{required files|see{prerequisites}}\index[general]{prerequisites|(}
Basic use of \oscola\ will require:
\begin{itemize}
\item The \textsc{biblatex} package. This is to be found in any modern
  \TeX\ distribution. You need version 2.0 (or higher).\index[general]{Biblatex!version 2.0 required}\index[general]{prerequisites!Biblatex}
\item The files \texttt{oscola.bbx}, \texttt{oscola.cbx} and
  \texttt{english-oscola.lbx}, installed where \LaTeX\ can find them.\index[general]{oscola.bbx@\texttt{oscola.bbx}}\index[general]{oscola.cbx@\texttt{oscola.cbx}}\index[general]{english-oscola.lbx@\texttt{english-oscola.lbx}}\index[general]{prerequisites!package files}
\item A working version of \textsc{biber}; this is also to be found in
  any modern \TeX\ distribution. You need version 1.0 (or higher).\index[general]{prerequisites!Biber}\index[general]{Biber!version 1.0 required}
\end{itemize}

If you intend to use the facilities that the package offers to produce
automatic tables of cases and legislation, you will also need:
\begin{itemize}
\item Certainly \textsc{makeindex}\index[general]{prerequisites!Makeindex}\index[general]{Makeindex!required if indexing} (which comes with any \TeX\ distribution).
\item Probably also \textsc{splitindex},\index[general]{prerequisites!Splitindex}\index[general]{Splitindex!recommended} since if you are producing
  more than one or two indexes this is likely to be essential. It
  comes with \TeX\ distributions; but I have found that you also need
  a working version of \textsc{Perl}.\index[general]{prerequisites!Splitindex!Perl} A Mac or Linux\slash Unix set up
  will almost certainly have this: a Windows machine will probably
  need to have it downloaded, but it's not hard to find.
\item The \textsc{imakeidx}\index[general]{Imakeidx!recommended}\index[general]{prerequisites!Imakeidx} package, which will either be in your
  \TeX\ distribution or easily obtainable.
\item If you want to use the index style file provided with \oscola,
  you will need to install it (\texttt{oscola.ist})\index[general]{oscola.ist@\texttt{oscola.ist}}\index[general]{prerequisites!package files} where
  \textsc{makeindex} can find it, for instance in the project
  directory.
\end{itemize}
\index[general]{prerequisites|)}

\subsection{Installation}

\index[general]{installation}
\index[general]{oscola.bbx@\texttt{oscola.bbx}}
\index[general]{oscola.cbx@\texttt{oscola.cbx}}
\index[general]{oscola.ist@\texttt{oscola.ist}}
\index[general]{english-oscola.ist@\texttt{english-oscola.ist}}
I would suggest installation in your (local) \TeX\ directory as follows:
\begin{description}
\item[oscola.bbx] With the bibliography style files in \texttt{.../\allowbreak tex/\allowbreak latex/\linebreak biblatex/\allowbreak bbx}.
\item[oscola.cbx] With the citation style files in \texttt{.../tex/latex/\linebreak biblatex/cbx}.
\item[english-oscola.lbx] With the language definition files in \texttt{.../tex/latex/\linebreak biblatex/lbx}.
\item[oscola.pdf] With documentation in a suitable directory under\linebreak \texttt{.../doc/latex}.
\item[oscola.ist] With the index style files in \verb|.../makeindex/latex|.
\item[oscola-examples.bib] With the sample bibliography files in a suitable directory under \verb|.../bibtex/bib|.
\end{description}

\section{Basic Use}

The \oscola\ package is not really a package, but a set of style files
specifically designed for \biblatex. To load them, therefore,
you simply load \biblatex, specifying the \oscola\ style:\index[general]{loading style}\index[general]{Biblatex!loading package}\index[general]{usepackage@\texttt{\textbackslash usepackage}}

\begin{bibexample}[loading]
\begin{verbatim}
\usepackage[style=oscola]{biblatex}
\end{verbatim}
\end{bibexample}

You will also need to identify one or more bibliography files:\index[general]{addbibresource@\texttt{\textbackslash addbibresource}}
\begin{bibexample}[files]
\begin{verbatim}
\addbibresource{mybib.bib}
\end{verbatim}
\end{bibexample}
\noindent Where, of course, \texttt{mybib.bib} is the name of your own
bibliographic database.

To make quotation marks work properly,\index[general]{prerequisites!csquotes}\index[general]{csquotes}\index[general]{quotation marks!ensuring correct style} you also need to load the \textsc{csquotes} package, with the style \verb|british|.
\begin{verbatim}
\usepackage[style=british]{csquotes}
\end{verbatim}

\subsection{Customization}

\index[general]{customization|(}The \oscola\ style is not highly customisable. This
is by design. The intention is to provide a reasonably complete and
accurate implementation of a particular set of conventions, rather
than a foundation on which a large variety of distinct styles for
legal citation could be built. It permits \emph{limited} customisation
in areas where \oscola\ itself suggests alternative possibilities (for
instance whether `ibid' is used), and for certain typographical
features which might legitimately be under the control of the user. On
the very few occasions where I have consciously departed from the
\oscolashort\ standard, my revisions will only be used if they are
specifically switched on.

To switch options, include them in the list of options when loading
\biblatex. So, for example: 
\begin{bibexample}[optionseg]
\begin{verbatim} 
\usepackage[style=oscola,
            eutreaty=alternative, 
            ibidtracker=false]{biblatex} 
\end{verbatim} 
\end{bibexample}
\noindent will load \biblatex\ with the \oscola\ style, with the alternative option
for printing short references to the EU treaty, and without making use
of `ibid' in successive citations.

\begin{description}
\item[\texttt{eutreaty}]\index[general]{customization!EU treaties,
  shortened
  form}\index[general]{options!eu-treaty@\texttt{eu-treaty}}\index[general]{EU
  treaties!shortened citations} If\label{options} set to
  \texttt{alternative}, then shortened references to an EU treaty will
  be printed in the form `Art 23 TFEU' rather than `TFEU, Art 23'. The
  latter is, I think, strictly required by
  \oscolashort;\footcite[29]{oscola} but the alternative version is so
  common in writing by \textsc{eu} specialists that it seemed sensible
  to provide it as a possibility.
\item[\texttt{caseshorthands}]\index[general]{customization!shorthand
  case names}\index[general]{shorthands!case names,
  italic}\index[general]{options!casesshorthands@\texttt{casesshorthands}}
  As I read it, \oscolashort\ distinguishes between the short
  \emph{titles} of cases and abbreviations for cases (what we would
  call `shorthands') by printing short titles in italic, and
  abbreviations in roman type.\footcite[See `example 4'][15]{oscola}
  To my eye that looks odd. The option \texttt{caseshorthands=italic}
  will see to it that shorthand names for cases, like their short
  titles, will be printed in italic type.
\item[ibidstyle]\index[general]{customization!ibid}
                \index[general]{ibid!capitalization}
                \index[general]{options!ibidstyle@\texttt{ibidstyle}}
  The \oscolashort\ guide uses ibid in its lower-case form, even at
  the start of footnotes. If you prefer to use an upper-case form
  there, then set the \texttt{ibidstyle} option to \texttt{uc}. Please
  note that while this will work reliably with citations created using
  \verb|\footcite| or \verb|\autocite|, if you use the
  form\begin{verbatim}\footnote{\cite{ ... }}\end{verbatim}the need
  for a capital will not be `picked up' automatically. In such a case,
  if you are using capitals, you will need to insert the command
  \verb|\bibsentence| immediately before the \verb|\cite|. (This is a
  good reason to use \verb|\footcite| where you can.\label{ibidstyle})
\item[\texttt{ibidtracker}] \index[general]{customization!ibid}
  \index[general]{options!ibidtracker@\texttt{ibidtracker}}
  \index[general]{ibid!disabling use of} \textsc{Oscola} permits
  citations to use `ibid' where there is an unambiguous citation to a
  single authority in the preceding footnote, but it is not
  required.\footnote{`\ldots\ you \emph{can} generally use ``ibid''
    instead' (emphasis added) \cite[5]{oscola}.} By default,
  \oscola\ does use `ibid', where it is appropriate. But if you prefer
  to switch it off, you can do so using the \biblatex\ option
  \texttt{ibidtracker=false}.
\item[\texttt{citetracker}]
                 \index[general]{customization!shortened citations}
                 \index[general]{options!citetracker@\texttt{citetracker}}
                 \index[general]{references!shortened
    forms} \textsc{Oscola} permits you, in certain situations (notably
  when citing books, articles and cases within a reasonably short
  section) to use cross-references back to previous notes, instead of
  giving a full citation afresh, but it does not require you to do
  so.\footnote{\enquote{Note that it is also acceptable to give the
      full citation every time a source is cited, and some publishers
      and law schools may prefer this to the use of short forms.}
    \cite[5]{oscola}} If you want not to, you can do so using the
  \biblatex\ option \texttt{citetracker={}false}. The \oscolashort\
  standard requires full citations if the previous citation was
  in an earlier chapter, but \oscola\ will take care of that for
  you if you use an appropriate \texttt{refsegment} option.\footcite[57]{biblatex2}\index[general]{customization|)}

It is possible to use \enquote{ibid} without the
\verb|citetracker|. But turning the \verb|citetracker| off will affect
the treatment of shorthands, which will be disregarded.
\item[shortindex] By default, \oscola\ produces full citations for
  cases when printing indexes (see p \pageref{index:format} below). If
  you prefer a simpler format, where tables\slash indexes contain only
  the title and year of the case, use to option \texttt{shortindex}
  when loading \biblatex.
\end{description}

\section{The Bibliographic Database}

\subsection{Entry Types}

The \oscola\ style uses the standard bibliographical types such as
\texttt{@book}, \texttt{@article}, \texttt{@report} and so forth. It
also makes use of four uncommon types:
\begin{description}
\item[@jurisdiction] Used for cases from all jurisdictions.
\item[@legislation] Used for legislation, including the foundational \textsc{eu} treaties.
\item[@legal] Used for various other legal materials, including treaties and Hansard materials.
\item[@commentary] Used for certain works `of authority'.
\end{description}
The use of these types is described in more detail below.

In many cases, in addition to the entry types, \oscola\ requires you
to set the field \texttt{entrysubtype}, to give additional information
about the particular type of source. Again, details are given in the
individual sections below.

Another important aspect of the bibliographic database is the use of
the \texttt{keyword} field to include information about the particular
jurisdiction from which the source originates. This is described in
more detail in section \ref{countries}.

\subsection{Entry Fields}

In addition to the usual entry fields, \oscola\ invites or requires
the use of various non-standard fields, such as
\texttt{reporter}. Internally these are mapped to standard (or
semi-standard) fields such as \texttt{journaltitle} or \texttt{usera}
and the like; but it is not only convenient but better to use the
non-standard fields in this case, because it may assist if you need to
use your database with a package which takes a different approach to
mapping.

The particular fields used for each type of source are described in
more detail in the sections of this document that deal with particular
sources.

\subsection{Specific Guidance}

\subsubsection{Periods}

\index[general]{bibfile@\texttt{.bib} file!periods@periods}
\index[general]{bibfile@\texttt{.bib} file!full stop@full stops}
When entering your bibliographical data, it is advisable to use place
periods in abbreviations, even though \oscolashort\ does not require
them. Where they are inappropriate, \oscola\ will strip them
out. There are two reasons why it is better to use periods:
\begin{itemize}
\item It's quite easy to strip them out, but very hard to know where
  to add them. Other legal styles---for instance the
  Bluebook---require periods. As and when styles which use \biblatex\
  to typeset data in such styles are developed, a database that
  already has them will be easy to use, whereas one which has been
  created without them will require substantial changes.
\item In the case of names, the addition of periods is essential to
  make sure that initials are recognised as such.
\end{itemize}

So, for instance, enter:
\begin{bibexample}[fullpoints]
\begin{verbatim}
...
title = {Hamble Fisheries Ltd. v. L Gardner \& Sons Ltd.}
\end{verbatim}
\end{bibexample}

\subsubsection{Names\label{namelists}}

\index[general]{bibfile@\texttt{.bib} file!names}\index[general]{names!bibfile@in \texttt{.bib} file}
The \textsc{biber} program is pretty good at coping with names in a
variety of forms: `John Smith' or `Smith, John' will both end up being
correct. But be consistent.

There are three things you do need to watch:

\begin{itemize}
\item When giving the name of an institution as an author or editor, make sure that you enclose it in braces:
\begin{bibexample}[braces]
\begin{verbatim}
...
author = {{Department of Health}}
\end{verbatim}
\end{bibexample}

If you don't do this then as far as \textsc{biber} is concerned this will get turned into `Health D of' for the purposes of constructing bibliography, which is not what you want!
\item When giving lists of names, separate them with `and' not commas:
  \begin{bibexample}[namelistseg]
\begin{verbatim}
author = {First Author and Second Author and Third Author}
\end{verbatim}
  \end{bibexample}
\item When giving initials, make sure you add a full stop and space after each initial, so that \textsc{biber} knows it's dealing with an initial and not a (very short) name:\index[general]{bibfile@\texttt{.bib} file!initials}\index[general]{initials!bibfile@in \texttt{.bib} file}
  \begin{bibexample}[initialsstyle]
\begin{verbatim}
author = {Hart, H. L. A.}
\end{verbatim}
  \end{bibexample}
This will get \emph{printed} as HLA~Hart. If you forget the initials, it will end up being printed (wrongly) as H~L~A~Hart.
\end{itemize}

\subsubsection{Dates}

A
\label{dateformat}
\index[general]{bibfile@\texttt{.bib} file!dates}
\index[general]{dates!bibfile@in \texttt{.bib} file} date should be entered in the form
\textsc{yyyy-mm-dd}, and a range of dates in the form
\textsc{yyyy-mm-dd}:

\begin{description}[labelwidth=5cm, leftmargin=5.3cm]
\item[{1970-04-02}] 4 February 1970
\item[{1990-01-03/1991-10-30}] 3 January 1990--30 October 1991
\end{description}

In many cases it is unnecessary to enter a complete date (though it is
always permissible to do so). Ranges are rarely required for
compliance with \oscolashort.\footcite[They are needed only when referring
to sessions of Parliament: see][40]{oscola}

\subsubsection{Pages}

\index[general]{pages field@\texttt{pages} field}
\index[general]{bibfile@\texttt{.bib} file!pages field@\texttt{pages} field}
In most cases OSCOLA requires only the citation of the first page of
an article or case. But it is always permissible to include a full
range, separated by \verb|--|. In some cases the `page' is actually
not a page but a case number:\footcite[See, eg,][18]{oscola} it should
still be entered into the \texttt{pages} field.

\subsubsection{Pagination\label{pagination}}

\index[general]{pagination field@\texttt{pagination} field}
\index[general]{bib file@\texttt{.bib} file!pagination field@\texttt{pagination} field}
If your source is referred to by anything other than page number, it's a good idea to enter an appropriate \texttt{pagination} fields, to assist with pinpoint citations. Common examples are as follows:
\begin{description}
\item[paragraph] For references in the form `para 1'.
\item[{[]}] For references in the form [1] (used in cases).
\item[article] For references in the form `art 1'.
\item[section] For references in the form `s 1'.
\item[rule] For references in the form `r 1'.
\item[regulation] For references in the form `reg 1'
\end{description}

\subsubsection{Slashes}

\index[general]{slashes!bib file@in \texttt{.bib} file}
\index[general]{bibfile@\texttt{.bib} file!slashes}
\index[general]{slash@\texttt{\textbackslash slash}}
\index[general]{line breaking!avoiding problems}
It is quite common for legal citations to include slashes
(\texttt{/}), for instance in \textsc{ecj} case numbers, \textsc{eu}
legislation numbers, statutory instrument numbers and the like.

There are two ways you can enter these, and the choice you make matters. If you enter them directly
\begin{bibexample}[slashes]
\begin{verbatim}
2001/2312
\end{verbatim}
\end{bibexample}
\noindent then they will be treated as unbreakable. If you want them to be breakable, you should use the \verb|\slash| command instead:
\begin{bibexample}[slashes:2]
\begin{verbatim}
2001\slash 2312
\end{verbatim}
\end{bibexample}

Think carefully about this. Obviously, breaks are bad. But \emph{not
  breaking} can make it impossible to justify text. My general advice
is that you are `safe' using a simple \texttt{/} where the number are
short, but that for anything that is going to result in more than
about five characters of text, you are better off using the
\verb|\slash| command.

\subsubsection{Hyphenation}

\index[general]{hyphenation field}
\index[general]{line breaking!avoiding problems}
\index[general]{language}
Another difficulty you will find with making sure you get clean
linebreaks is hyphenation. You can do two things to help. First, add
explicit hyphenation in difficult cases. Secondly, if a citation is
predominantly in a non-English language, set the \texttt{hyphenation}
field in the entry to the relevant language. (You should, then, pass the option \texttt{hyphenation\allowbreak=\allowbreak babel} to \biblatex.)

\subsubsection{Quotations}

\index[general]{quotation marks}
\index[general]{enquote@\texttt{\textbackslash enquote}}
Occasionally a title contains material in quotation marks. To ensure
that whatever style of citation you are using this will get properly
printed, it is advisable to use the \verb|\enquote{}| macro from the
\textsc{csquotes} package rather than enter the quotation marks
directly. (Of course, for the \oscola\ style, which uses single
quotation marks and double within single (`Like ``this'' example'),
double quotation marks will do; but it's sensible to prepare your
database file to be usable with different styles. Example
\ref{craig05} demonstrates this.)

\section{Citation Commands}

\index[general]{citation commands|(}
As the standard makes clear, `\textsc{oscola} is a footnote style: all
citations appear in footnotes'.\footcite[3]{oscola} The \oscola\
package assumes you are following this approach.

The basic commands for citation are \verb|\cite| and
\verb|\footcite|. These work as set out in the \biblatex\
documentation.
\index[general]{cite@\texttt{\textbackslash cite}}
The format for \verb|\cite| is:
\begin{center}
\verb|\cite|$\langle$[\emph{prenote}]$\rangle \langle$[\emph{postnote}]$\rangle$\verb|{|\emph{label}\verb|}|
\end{center}
The \emph{prenote} is any material you want printed before the
citation (such as `See').\footnote{There is one time---when dealing
  with court rules and practice directions---when the \oscola\ style
  abuses the pre-note for special reasons. See section
  \ref{courtrules}.}
The \emph{postnote} is anything you want
printed after it -- usually the page or pages to which you are
referring. So long as it looks like a set of `pages', the package will
include any label or labels that ought to be added,\footnote{Based on the \texttt{pagination} field of the entry.} such as `para' or
`s' or `col', and (where possible) it is usually best to let it do
that (because it helps make sure the index is correct).

\begin{description}
\item[\egcite{\{oscola\}}]          \cite{oscola}
\item[\egcite{[3]\{oscola\}}]       \cite[3]{oscola} 
\item[\egcite{[See][3]\{oscola\}}]  \cite[See][3]{oscola}
\item[\egcite{[ch~1]\{oscola\}}]    \cite[ch~1]{oscola}
\end{description}

The \verb|\footcite| command works in just the same way, except that
it puts the whole of the citation (including the pre- and post-notes)
into a fotnote.
\index[general]{footcite@\texttt{\textbackslash footcite}}
You can also use the \verb|\cite| command in a
footnote, so \verb|\footcite[12]{oscola}| will usually produce the same output as
\verb|\footnote{\cite[12]{oscola}.}| But the \verb|\footcite| version usually leaves your source code easier to read.\footnote{If you are using \texttt{ibidstyle=uc} to have capitalized `Ibid' references, there is a further reason to use it. See page \pageref{ibidstyle}.}

If you prefer, you can use \verb|\autocite|, which will work in
practice more or less as \verb|\footcite| does.
\index[general]{autocite@\texttt{\textbackslash autocite}}
One of the best things about
\verb|\autocite| is that it is quite `clever' in the way that it moves
punctuation and spaces to make sure footnote references are correctly
placed.

Often you want multiple citations on one occasion. If that is the
case, you can use the \verb|\multicites| and \verb|\multifootcites|
commands, which are described in more detail in the \biblatex\
documentation.\index[general]{multicites@\texttt{\textbackslash{multicites}}}

The package supports the use of the \verb|\textcite| command, but it is not fully developed. It makes most sense for cases, where \verb|\textcite{boardman}| will produce a citation with the name of the case in the text, and the reference in the footnote, so that one can refer to \textcite{boardman} in running text. For books and articles it provides the author's name. For other material it is not guaranteed to produce useful results.


\index[general]{citation commands|)}

\subsection{Multiple postnotes}
\index[general]{citation commands!multiple postnotes}
\index[general]{English Reports}
\index[general]{rules of court}
\index[general]{parallel citations} In two cases, \oscola\ uses `multiple' postnotes. Consider the following citation:
\begin{center}
\cite[92|996]{henly28}
\end{center}
As you can see there are two separate `pinpoints'---{}page 92 (in the nominate report) and page 996 (in the English Reports){}---which need to be separately handled. Similar cases occur in relation to a citation to a rule of court such as
\begin{center}
\cite[11|6]{rsc}
\end{center} 

In such cases you need to divide the \texttt{postnote} into two. You do this by splitting the parts with the pipe character (\verb=|=). So the citation to \citetitle{henly28} was produced by
\begin{bibexample}[doublepostnote]
\begin{verbatim}
\cite[92|996]{henly28}
\end{verbatim}
\end{bibexample}

\section{Bibliography}
\index[general]{bibliography!printing|(}
\index[general]{bibliography!file|see{\texttt{.bib} file}}

In order to produce the bibliography, you use the usual
\verb|\printbibliography|.
\index[general]{printbibliography@\texttt{\textbackslash printbibliography}}
But there are likely to be complications;
because as it stands this will print \emph{everything} you have cited
including cases and statutes. You probably don't want this: the
likelihood is that you will want to put cases and statutes in indexed
tables, as described in section \ref{indexing}.

\index[general]{bibliography!printing!type option@\texttt{type} option}
\index[general]{bibliography!printing!nottype option@\texttt{nottype} option}
So you need to exclude the unwanted material from the bibliography. To do that, make use either of the \texttt{type} or \texttt{nottype} options from \biblatex. A pretty typical set up would be along these lines:
\begin{bibexample}[bibliography]
\begin{verbatim}
\printbibliography[nottype=commentary,
                   nottype=jurisdiction,
                   nottype=legislation,
                   nottype=legal]
\end{verbatim}
\end{bibexample}
This will exclude cases, legislation, treaties, commentaries, and
Hansard references from the bibliography.

Of course you can also use other options, such as your own keywords
and so on, to include or exclude entries from the bibliography. You
can also, if you wish to do so, set the option \verb|skipbib=true| on
an entry that you never want included. The package does not, however,
set this automatically, because I find it useful to be able to produce
a full bibliography when working with draft documents, even if it is
trimmed down once writing and proofing is complete.

Note that in the bibliography, first names are abbreviated in all cases to initials (as \oscolashort\ requires).\index[general]{bibliography!printing|)}

\section{Indexing\label{indexing}}
\index[general]{indexing|(}
\index[general]{table of cases|see {indexing}}

Although legal works usually have only one `index' as such, they can
have (and \oscolashort\ requires them to have) several
tables which are, in effect, indexes. \oscola\ helps to create those,
as far as possibly automatically.

The basic idea is this. When you cite a case or statute (or, in fact)
any other kind of work, \oscola\ tries to write a suitable piece of
information in an external file which will, with the help of another
\index[general]{Imakeidx!recommended}
\index[general]{Splitindex!recommended}
package (\textsc{imakeidx} is the one I recommend) and an external
program or programs (I recommend \textsc{splitindex})\footnote{Why
\textsc{imakeidx} and \textsc{splitindex}? I like \textsc{splitindex}
because otherwise you are likely to be bitten by the fact that \LaTeX\
can only write a number of auxiliary files. I like \textsc{imakeidx}
because it provides a convenient method of using
\textsc{splitindex}. You cannot, in fact, use \textsc{splitindex}
directly with \oscola.} produce the necessary tables.

The complication is in the word `tries'. The difficulty is that it's
not obvious how many indexes you want. \textsc{Oscola} leaves you
with a fairly wide choice. A simple paper might require no more than a
table of legislation and a table of cases. A complex book might
require several different tables, including separate tables for
primary and secondary legislation, \textsc{eu} treaties, English,
\textsc{eu}, international and other cases. 

\subsection{Stage One: Turn Indexing On}

\index[general]{indexing!enabling}
\index[general]{options!indexing@\texttt{indexing}}
The first stage is to turn indexing on. To do that you need to do two
things: first, specify \texttt{indexing=cite} as an option when you
load \biblatex.
\begin{center}
\verb|\usepackage[...indexing=cite...]{biblatex}|
\end{center}

Next load a suitable indexing package: I'm going to assume \textsc{imakeidx}:\footnote{The package works with both \textsc{index} and \textsc{imakeidx}. It is not currently compatible with \textsc{multind}.}
\index[general]{imakeidx!recommended}
\index[general]{index (package)!compatible with style}
\index[general]{multind!incompatible with style}
\begin{center}
\verb|\usepackage{imakeidx}|
\end{center}
You should consult the \textsc{imakeidx} documentation for details of
the options you might pass to \textsc{imakeidx}. But if you are
planning to use \textsc{splitindex} you should always include the
|splitindex| documentation.

\subsection{Stage Two: Hook Up the Indexes}

\index[general]{indexing!setting up indexes|(}
Next you need to `hook up' your indexes. Unknown to you, \oscola\ is
creating a bunch of potential indexes every time it runs, but as
things stand you they are all directed into an index file called,
appropriately enough `trash', which is created for you. What you need
to do is `divert' them to useful indexes. Your task is to `connect'
these virtual indexes to particular index files. It's possible (indeed
common) to connect more than one `virtual' index to a single index
file.

First decide what indexes you are going to need. Let's suppose you
want two: a table of cases and a table of legislation. You need to set
up those indexes for \textsc{imakeidx}. In your preamble:
\index[general]{makeindex@\texttt{\textbackslash makeindex}}

\begin{bibexample}[makeindex]
\begin{verbatim}
\makeindex[name=cases, title={Table of Cases}]
\makeindex[name=legislation, title={Table of Legislation}]
\end{verbatim}
\end{bibexample}

Now you need to associate certain entries or entrytypes with your
indexes. You can do that on an entry-by-entry basis by specifying the
\index[general]{tabulate field@\texttt{tabulate} field}
\index[general]{bibfile@\texttt{.bib} file!tabulate field@\texttt{tabulate} field}
index name in the \verb|tabulate| field (see page \pageref{tabulate}). But it's much more convenient,
usually, to make use of one of the pre-installed groupings.

\index[general]{DeclareIndexAssociation@\texttt{\textbackslash DeclareIndexAssociation}}
\index[general]{indexing!setting up indexes!DeclareIndexAssociation@\texttt{\textbackslash DeclareIndexAssociation}}
To do this, use the
\verb|\DeclareIndexAssociation{|\angledtext{category}\verb|}{|\angledtext{index}\verb|}|
macro. It\linebreak takes two arguments. The first is the category of entries
you want included in the index, selected from the list in table
\ref{categories}. The second is the index you want to receive the data
in relation to that category of entry. So, for instance:
\begin{bibexample}[declareformat]
\begin{verbatim}
\DeclareIndexAssociation{gbcases}{cases}
\DeclareIndexAssociation{encases}{cases}
\DeclareIndexAssociation{sccases}{cases}
\DeclareIndexAssociation{nicases}{cases}
\end{verbatim}
\end{bibexample}
\noindent
would mean that UK, English, Scottish and Northern Irish cases are all
indexed in the `cases' index.

\begin{table}
\centering
\small
\index[general]{indexing!virtual indexes, list of}
\begin{tabular}{lll}
\toprule
source type                          & hook                  & order \\
\midrule
UK cases                             & \texttt{gbcases}      & alphabetical\\
English cases                        & \texttt{encases}      & alphabetical\\
Scottish cases                       & \texttt{sccases}      & alphabetical\\
Northern Irish cases                 & \texttt{nicases}      & alphabetical\\
EU cases                             & \texttt{eucases}      & alphabetical \\
                                     & \texttt{eucasesnum}   & by case number \\
ECHR cases                           & \texttt{echrcase}     & alphabetical \\
ECHR Commission decisions            & \texttt{echrcasescomm}& alphabetical \\
International cases                  & \texttt{pilcases}     & alphabetical \\
Other cases                          & \texttt{othercases}   & alphabetical \\
UK primary legislation               & \texttt{gbprimleg}    & alphabetical \\
UK draft legislation                 & \texttt{gbdraftleg}   & alphabetical \\
English draft legislation            & \texttt{gbdraftleg}   & alphabetical \\
English primary legislation          & \texttt{enprimleg}    & alphabetical \\
Scottish primary legislation         & \texttt{scprimleg}    & alphabetical \\
Welsh primary legislation            & \texttt{cyprimleg}    & alphabetical \\
Northern Irish primary legislation   & \texttt{niprimleg}    & alphabetical \\
UK secondary legislation             & \texttt{gbsecleg}     & alphabetical \\
English secondary legislation        & \texttt{ensecleg}     & alphabetical \\
Rules of court                       & \texttt{enroc}        & alphabetical \\
Scottish secondary legislation       & \texttt{scsecleg}     & alphabetical \\
Welsh secondary legislation          & \texttt{cysecleg}     & alphabetical \\
Northern Irish secondary legislation & \texttt{nisecleg}     & alphabetical \\
EU Treaties                          & \texttt{eutreaty}     & alphabetical \\
ECHR Treaty                          & \texttt{echrtreaty}   & alphabetical \\
EU regulations                       & \texttt{euregs}       & numeric \\
EU directives                        & \texttt{eudirs}       & numeric \\
EU decisions                         & \texttt{eudecs}       & numeric \\
Other treaties                       & \texttt{piltreaty}    & alphabetical \\
English parliamentary material       & \texttt{gbparltmat}   & alphabetical \\
EU official documents                & \texttt{euoffdoc}     & alphabetical \\
Commentaries                         & \texttt{commentaries} & alphabetical 
                                                               by author \\
Names                                & \texttt{namesindex}   & alphabetical\\
\bottomrule
\end{tabular}
\caption{Virtual Indexes\label{categories}}
\end{table}

Although\label{tabulate} the various hooks provided automatically
should often be enough, it is possible to define your own indices, if you
wish. To do so, set the \texttt{tabulate} field of any entry to the
name of the index that is to be used for it. So, for instance, if you
wanted US cases to be placed in an index called `uscases', you could
set each US case as follows:
\begin{bibexample}[tabulation]
\begin{verbatim}
...
tabulate = {uscases},
...
\end{verbatim}
\end{bibexample}
There is a difference between an index set automatically and one set
using the \verb|tabulate| field. The automatic indexes are
virtual. They can be redirected, and will (unless you associate them)
be automatically directed to the trash index. But an index listed in
the \verb|tabulate| field is passed \emph{directly} to your index
command, and so if you set a tabulate field you need to make sure that
such an index is created, or you will see
errors.\index[general]{indexing!setting up indexes|)}


\subsection{Stage Three: Process and Print Indexes}

\index[general]{indexing!processing and printing|(}
The \textsc{imakeidx} package is, in theory, able to call external
programs in order to create indexes without the need for multiple
compilations. Unfortunately, it can only do so effectively if the
indexes are being printed right at the end of the document. This is
not the usual position for them in legal work.

If you do want your tables\slash indexes at the end of your text, then
you just use the ordinary \textsc{imakeidx} \verb|\printindex| command
\index[general]{printindex@\texttt{\textbackslash printindex}} in the
usual way. So a typical usage would look something like:
\begin{bibexample}[printindex:1]
\begin{verbatim}
\printindex[cases]
\end{verbatim}
\end{bibexample}

If you are operating \textsc{imakeidx} automatically, this will
generate fresh and accurate indexes on each run. If you are not using
the automatic facilities that package offers, you will need to run
\textsc{splitindex} or \textsc{makeindex} before finally re-running
\LaTeX\ to print the final index.

If you want your tables\slash indexes at the beginning of your text
(as is common in legal work), you have to take a slightly different
\index[general]{printindexearly@\texttt{\textbackslash printindexearly}}
tack. Instead of the \verb|\printindex| command offered by
\textsc{imakeidx}, you need to use the command
\verb|\printindexearly|. This functions just like \verb|\printindex|
(for instance it takes the same arguments), but it does not attempt to
generate an index file automatically: it cannot do so, because at an
early stage in the document's typesetting the relevant information is
not available.

In such a case, you will necessarily have to run \textsc{splitindex}
or \textsc{makeindex} yourself. So the workflow will be:
\begin{itemize}
\item Run \LaTeX\ on the source code.
\item Run \textsc{biber}.
\item Run \LaTeX\ again, twice (to make sure all cross-references are properly generated).
\item If you are using \textsc{splitindex}, run it on the master index
  file to generate the actual indexes ready for typesetting. If you
  are using \textsc{makeindex}, run it on each of the individual
  \texttt{.idx} files to generate the actual indexes ready for typesetting.
\item Run \LaTeX\ at least once more to typeset those indexes and
  generate the final copy.
\end{itemize}
\index[general]{indexing!processing and printing|)}

In most cases you will want tables of cases and legislation to be
formatted with dot leaders, rather than in the default style. To
achieve that you need to use the \verb|oscola.ist| style. There are
two ways of doing this:
\begin{itemize}
\item If you are using \textsc{makeindex}, then run it on each of the verb|.idx| files using the |-s oscola| option:
\begin{center}
  \texttt{makeindex} \angledtext{indexname} \texttt{-s oscola}
\end{center}
\item If you are using \textsc{splitindex}, run it with the option
  \verb|-m|, and then process each of the resulting sub-indices that
  you want formatted with leaders with the option \verb|-s oscola|.
\item Alternatively, run \textsc{splitindex} with the option \verb|-- -s oscola|, which will run \textsc{makeindex} on each file with the requisite style.
\end{itemize}

The only real annoyance is when you have multiple indexes and want
different formats. In such a case, the solution is to generate most of
the indexes as set out above, and then run \textsc{makeindex} on the
`oddballs' with appropriate options.

\subsection{Some finer points}

The basic method for producing indexes has been explained. There are,
however, a few subtle points that you may well encounter, especially
if your work is complicated.

\subsubsection{Titles}

\index[general]{indexing!adjusting what is printed}
\index[general]{indextitle field@\texttt{indextitle} field}
\index[general]{bibfile@\texttt{.bib} file!indextitle field@\texttt{indextitle} field}
Most indexes are based on titles. But sometimes you want a slightly
different title in the index than in the text. A typical example would
be a case name that begins `Re'. You want this in textual citations as
\emph{Re Matter}, but in the index you want it listed as \emph{Matter,
  Re}. To do this you will need to set the \textsc{indextitle} field
to the form you want used in the index.

\subsubsection{Suppressing Indexing}

\index[general]{indexing!suppressing}
\index[general]{suppress indexing!for source}
There may be occasions when you want to suppress the indexing either
of a particular entry, or of a particular citation.\footnote{Of course, to suppress it in general, just don't use the \texttt{indexing} option at all!}

To suppress indexing of a particular entry whenever it is cited, set
its \texttt{tabulate} field to `trash'. This effectively means that it
will never find its way into any real index.\index[general]{tabulate field@\texttt{tabulate} field}
\index[general]{bibfile@\texttt{.bib} file!tabulate field@\texttt{tabulate} field}

\index[general]{suppress indexing!for citation}
\index[general]{DNI@\texttt{\textbackslash DNI}}
To suppress indexing of a particular citation, put the command
\verb|\DNI| (which stands for `Do Not Index') immediately before the
citation. This is mostly useful when you are intending to provide
your own indexing entry for the citation, and need to suppress the one
that would be automatically provided.

\subsubsection{Adding index entries\label{trickyindexing}}

\index[general]{indexing!manual additions|(}
\index[general]{indexing!adding entries manually|(}
Finally there are two commands you can use to insert index entries yourself.

\index[general]{indexonly@\texttt{\textbackslash indexonly}}
The first is \verb|\indexonly|. This will simply insert an index
citation formatted exactly as if you had cited a particular source,
but without printing anything.

So, for instance, \verb|\indexonly[2]{ucta77}| will insert an index
entry for section 2 of the Unfair Contract Terms Act 1977 into the
index.

\index[general]{indexing!index command@\texttt{\textbackslash index} command}
The second is to use the \verb|\index| command directly. This can be
useful for more sophisticated things. You use it exactly as you would
use it on any other occasion. This is likely to be a rare occurrence:
its most common use will be for inserting a cross-reference (for
instance to a ship's name).

\index[general]{indexing!cross references}
\index[general]{index@\texttt{\textbackslash index}}
\index[general]{ships' names!adding to index}
For example, suppose you have the following case (which also
shows the use of additional reports for listing in the index):
\begin{bibexample}[antaios85]
\begin{verbatim}
@jurisdiction{antaios85,
  title          = {Antaios Compania Naviera S.A. v. 
                    Salen Rederierna A.B. (The Nema)},
  shorttitle     = {The Nema},
  date           = {1985},
  reporter       = {A.C.},
  pages          = {191},
  court          = {H.L.},
  additionalreports = {[1984] 3 WLR 592 and (1984) 128 SJ 564 
                       and [1984] 3 All ER 229 and 
                       [1984] 2 Lloyd's Rep 235},
  keywords       = {gb},
}
\end{verbatim}
\end{bibexample}

This will produce an index entry of:
\begin{quote}
\citeinindex{antaios85}
\end{quote}\label{index:format}

But you also need to add an entry for its ship's name. To do that, assuming the relevant index is called \texttt{ukcases}, you would simply write:
\begin{verbatim}
\index[ukcases]{Nema, The@\emph{Nema,} The|see{Antaios 
                Compania Naviera SA v Salen Rederierna SA}}
\end{verbatim}
and the relevant entry would be produced.
\index[ukcases]{Nema, The@\emph{Nema,} The|see{Antaios Compania Naviera SA v Salen Rederierna SA}}\indexonly{antaios85}

\subsubsection{Legislation}

\index[general]{indexing!legislation}
\index[general]{legislation!indexing}
These techniques really come into their own when dealing with complex
references to statutes or treaties.

To see the problem, you have to understand how \oscola\ handles a
citation like \verb|\cite[2(1)]{ucta}|. First it tries to break down
the postnote (the pinpoint citation) into two parts: the first part
(equivalent to section) and the second (the subsection). Having done
that, it then inserts and entry into the index with a `depth' of
three: UCTA, s 1, (1). This is normally what one wants.

It copes perfectly well, also, with lists of references. If you had
entered a command like \verb|\cite[1(2), 2(3)]{ucta}| the package will
put in two references: one for section 1, subsection 2, and one for
section 2, subsection 3.

However, it has problems in two cases. First, with ranges. If you give
a range like \verb|\cite[1--3(1)]{ucta}|, the package has no way of
knowing how many intermediate parts are implicitly cited. In that
case, it is lazy: it automatically adds \emph{only the first} citation
to the index (in that case, section 1). If you want other parts added,
you should do it manually. For instance with:
\begin{bibexample}[adding:index]
\begin{verbatim}
\cite[1--3(1)]{ucta}  % Prints citation, indexes s 1
\indexonly[2]{ucta}   % Adds s 2 to the index 
\indexonly[3(1)]{ucta}% Adds s 3(1) to the index 
\end{verbatim} \end{bibexample}

The second, and more serious problem, is with references that are out
of the ordinary `run', for instance to schedules and
paragraphs. There's a good chance these will end up being sorted
incorrectly, and if so you really have no choice but to insert an
index entry entirely manually.

\index[general]{index@\texttt{\textbackslash index}}
This is a bit trickier. The basic pattern you will need is as follows:
\begin{bibexample}[rawindex:1]
\verb|\index[|\angledtext{index}\verb|]{%|\\
\angledtext{index title}\verb|%|\\
\verb|@\citeinindex {|\angledtext{key}\verb|}%|\\
\verb|!|\angledtext{sorting for sub level}\verb|@|\angledtext{printing for first level}\verb|%|\\
\verb|!|\angledtext{sorting for second sub level}\verb|@|\angledtext{printing for second sub level}\verb|}|
\end{bibexample}

As to the parts:
\begin{description}
\item[index title] The title as used to sort the index, eg
  \texttt{Human Rights Act 1998}
\item[\textbackslash citeinindex \{\}] This will print the title, and
  any other details, correctly for the index. Please observe the space
  between the command name and the brackets, which matters to ensure
  proper sorting.
\item[sorting] These are the keys for sorting the level. For instance,
  if you want to insert a reference to section 4A, and make sure it
  gets sorted before section 4, you might give the reference here as
  3.
\item[printing] This is what will get printed as the label at that `level'.
\end{description}

So, for instance
\begin{bibexample}[rawindex:2]
\begin{verbatim}
\index[statutes]{%
  Human Rights Act 1998%
  @\citeinindex {hra98}%
  !100@Sched 1%
  !1@para 1}
\end{verbatim}
\end{bibexample}
\noindent
would insert a reference to Human Rights Act 1998, Sched 1, para 1,
which would appear `as if' it were section 100 of the Act, i.e. after
all the other sections.
\index[general]{indexing!manual additions|)}
\index[general]{indexing!adding entries manually|)}

\section{Cases}
\index[general]{cases!in general|(}

\subsection{Fields}

\index[general]{cases!fields|(}
Cases should be entered into the bibliography database using the
\texttt{@jurisdiction} entrytype. Use the following fields:

\begin{description}
\item[\texttt{title}]
\index[general]{title field@\texttt{title} field!cases@in cases}
\index[general]{bibfile@\texttt{.bib} file!title field@\texttt{title} field}
Mandatory. The full title of the case, as it
  should appear on first citation. Any periods (full stops) used in this field are stripped out automatically.
\item[\texttt{shorttitle}]
\index[general]{shorttitle field@\texttt{shorttitle} field!cases@in cases}
\index[general]{bibfile@\texttt{.bib} file!shorttitle field@\texttt{shorttitle} field}
Optional. A shorter title of the case which
  will \emph{silently} replace the full title on second and subsequent
  citations in any reference section. So, for instance, \emph{R v
    Caldwell} would have its \texttt{shorttitle} set to
  \texttt{Caldwell}. Use this field for abbreviated case names which
  are those that any lawyer would naturally understand without
  explanation.
\item[\texttt{shorthand}]
\index[general]{shorthand field@\texttt{shorthand} field!cases@in cases}
\index[general]{bibfile@\texttt{.bib} file!shorthand field@\texttt{shorthand} field}
Optional. A shorter title of the case which
  will be \emph{introduced} the first time the case is cited in any
  reference section, and thereafter used in place of the full title,
  and will be listed in any table of abbreviations. Use this field
  sparingly.
\item[\texttt{date}]
\index[general]{date field@\texttt{date} field!cases@in cases}
\index[general]{bibfile@\texttt{.bib} file!date field@\texttt{date} field}
\index[general]{bibfile@\texttt{.bib} file!dates}
Mandatory. The date which it is appropriate to
  use in the citation of the report you will be using, which may or
  may not be the actual date of the decision. Only the year is
  required (except for some unreported cases, where \oscolashort\
  requires a full date). See section \ref{dateformat}
  for information about how to enter dates.
\item[\texttt{origdate}]
\index[general]{origdate field@\texttt{origdate} field!cases@in cases}
\index[general]{bibfile@\texttt{.bib} file!origdate field@\texttt{origdate} field}
Optional. The date of the decision itself if
  it is different from the date being given in the \texttt{date}
  field, \emph{and sufficiently important to matter}.
\item[\texttt{number}]
\index[general]{number field@\texttt{number} field!cases@in cases}
\index[general]{bibfile@\texttt{.bib} file!number field@\texttt{number} field}
\index[general]{neutralcite field@\texttt{neutralcite} field}
\index[general]{bibfile@\texttt{.bib} file!neutralcite field@\texttt{neutralcite} field}
Optional, but sometimes required for correct
  citation. This field is used for any number which forms an essential
  part of the case's citation. Its exact use varies depending on the
  type of case. For English, Scottish, Australian and Canadian cases it is the full neutral
  citation, if there is one. For \textsc{eu} cases it is the case
  number; for US cases the docket number (Note that \texttt{neutralcite} may also be used: it
  functions as a synonym for number.)
\item[\texttt{keywords}]
\index[general]{keywords field@\texttt{keywords} field!cases@in cases}
\index[general]{bibfile@\texttt{.bib} file!keywords field@\texttt{keywords} field}
Multiple keywords can be given; but the
keywords field\linebreak should always include at least one `country'
indication. For detail see section \ref{countries}.
\item[\texttt{court}]
\index[general]{court field@\texttt{court} field}
\index[general]{bibfile@\texttt{.bib} file!court field@\texttt{court} field}
\index[general]{institution field@\texttt{institution} field!cases@in cases}
\index[general]{bibfile@\texttt{.bib} file!institution field@\texttt{institution} field}
Optional. This gives the court which decided the
  case. It is normally printed only if required.\footnote{Not quite
    true. The package is pretty good at working out whether it is
    required for English and European citations; but it `knows less'
    about non-English citations, and will generally use any
    information it is given when printing them.} For instance, in an
  English case which has a neutral citation the information will not
  be printed, since the neutral citation provides it anyway. Similarly
  with an \textsc{eu} case. (Note that internally \texttt{court} is a
  synonyn for the `institution' field, which you may use if you
  prefer.)
\item[\texttt{reporter}]
\index[general]{reporter field@\texttt{reporter} field!cases@in cases}
\index[general]{bibfile@\texttt{.bib} file!reporter field@\texttt{reporter} field}
\index[general]{journaltitle field@\texttt{journaltitle} field!cases@in cases}
\index[general]{bibfile@\texttt{.bib} file!journaltitle field@\texttt{journaltitle} field}
Mandatory, except for unreported
  cases. The title of the series of reports to which you are
  citing. (The \texttt{reporter}  field is a synonym for
  \texttt{journaltitle}, which you may use if you prefer.)
\item[\texttt{series}]
\index[general]{series field@\texttt{series} field!cases@in cases}
\index[general]{bibfile@\texttt{.bib} file!series field@\texttt{series} field}
Optional. The series of the reports in
  question.
\item[\texttt{volume}]
\index[general]{volume field@\texttt{volume} field!cases@in cases}
\index[general]{bibfile@\texttt{.bib} file!volume field@\texttt{volume} field}
The volume of the reports in which the case
  appears. If there is more than one volume then, of course, the field
  is essential.
\item[\texttt{pages}]
\index[general]{pages field@\texttt{pages} field!cases@in cases}
\index[general]{bibfile@\texttt{.bib} file!pages field@\texttt{pages} field}
If the case is reported, you need to give a page
  reference. To comply with \oscolashort\ you only need to give
  the first page. A `completist' might want to make sure that all the
  pages are in the database, in case some other style is used
  later. That's fine: \oscola\ will simply ignore the extra
  information. (But it's probably a waste of time: I don't know
  \emph{any} citation style in law that uses this information.)
\item[\texttt{pagination}]
\index[general]{pagination field@\texttt{pagination} field!cases@in cases}
\index[general]{bibfile@\texttt{.bib} file!pagination field@\texttt{pagination} field}
The type of pagination the report uses: see
  section \ref{pagination} above for details: it defaults to page for
  all cases other than \textsc{eu} and \textsc{echr} cases, where it
  defaults to paragraph.
\item[\texttt{options}] 
\index[general]{options field@\texttt{options} field!cases@in cases}
\index[general]{bibfile@\texttt{.bib} file!options field@\texttt{title} field}
\index[general]{options field@\texttt{options} field!year-essential@\texttt{year-essential}}
Only required in some cases. The option
  that you may need is \texttt{year-essential=true}
  for cases where it is not obvious that the year is an essential
  part of the citation, but it is. This is explained further in
  section \ref{yearoptional} below.
\item[\texttt{note}]
\index[general]{note field@\texttt{note} field!cases@in cases}
\index[general]{bibfile@\texttt{.bib} file!note field@\texttt{note} field}
An optional short note to be printed following the case
  citation. This is really intended for adding the word `note' in
  those cases where the \emph{Law Reports} contain a note rather than
  a report of a case.
\item[location]
\index[general]{location field@\texttt{location} field!cases@in cases}
\index[general]{bibfile@\texttt{.bib} file!location field@\texttt{location} field}
The place where a court that decided a case sits. This
is not used for English cases. It is used in Canadian, US and
Australian cases where the location of the court may matter.
\item[\texttt{parreporter} etc]
\index[general]{parallel citations}
\index[general]{English Reports}
\index[general]{parreporter field@\texttt{parreporter} field}
\index[general]{bibfile@\texttt{.bib} file!parreporter field@\texttt{parreporter} field}
\index[general]{parvolume field@\texttt{parvolume} field}
\index[general]{bibfile@\texttt{.bib} file!parvolume field@\texttt{part} field}
\index[general]{parseries field@\texttt{parseries} field}
\index[general]{bibfile@\texttt{.bib} file!parseries field@\texttt{parseries} field}
\index[general]{parpages field@\texttt{parpages} field}
\index[general]{bibfile@\texttt{.bib} file!parpages field@\texttt{parpages} field}
Additional fields \texttt{parreporter},
\texttt{parseries}, \texttt{parvolume} and \texttt{parpages} are used
to give citation details to a second report, where it is necessary to
cite more than one reporter. See section \ref{parallel:reports} for
further information.
\item[\texttt{additionalreports}]
\index[general]{indexing!additional reports}
\index[general]{parallel citations}
\index[general]{additionalreports field@\texttt{additionalreports} field}
\index[general]{bibfile@\texttt{.bib} file!additionalreports field@\texttt{additionalreports} field}
This field is used to provide a list
of additional reports. The list is intended for use only in the index
or table of cases, and can be used if you want to provide
comprehensive details about all possible reports of a case. Items should be separated by `and': see example \ref{antaios85}. By default, a comma is printed between items. If you prefer some other delimiter, redefine \verb|\extracitedelim|.\index[general]{extracitedelim@\texttt{\textbackslash extracitesdelim}}
\end{description}
\index[general]{cases!fields|)}


Presented in that way, the information may seem rather daunting, but
it's actually fairly simple in practice.\index[general]{cases!fields!overview|(}
\begin{itemize}
\item The \texttt{title} and \texttt{shorttitle} fields are used to
  give the full title of the case (for first citation) and,
  optionally, a shortened title for subsequent citation. The
  \texttt{shorthand} field is available for those rare cases where you
  want to use a short title that is not obvious.
\item The \texttt{neutralcite} (or \texttt{number}) holds the case
number or neutral citation, for cases where
  there is one.
\item The \texttt{date}, \texttt{volume}, \texttt{series},
  \texttt{reporter} and \texttt{pages} fields hold the various
  parts of a reference to a report. In cases where you will need to
  include a second parallel citation, the \texttt{parvolume},
  \texttt{parseries}, \texttt{parreporter} and \texttt{parpages} field
  hold the equivalent infomration for the parallel report. (It is
  assumed that the date field can remain unchanged: I know of no case
  where parallel citations would take a form for which this is problematic.)
\item The \texttt{court}, \texttt{location} and \texttt{note} fields contain
  additional information about the case, sometimes useful for the
  reader.
\item The \texttt{additionalreports} field is intended to provide a
comprehensive list of alternative reports, only for citation in the index.
\item The \texttt{pagination}, \texttt{options} and \texttt{keywords}
  fields hold `metadata' that enables the \oscola\ package to get the
  citation correct.
\end{itemize}
\index[general]{cases!fields!overview}

It is important to appreciate that although the style cannot create
information you do not supply (1) it will do the best with the
information you do supply (so, for instance, if you don't supply a
\texttt{reporter} it will assume the case is unreported, and
format a citation accordingly) and (2) it will never use information
that the \oscolashort\ standard doesn't require, even if you
supply it. So you don't need to worry about whether to include
information about the institution deciding the case alongside a
neutral citation. If it's not needed (as in fact it
isn't\footcite[16]{oscola}) the package will simply not use it.

\subsection{[Year] or (Year)}

\index[general]{dates!whether year essential to citation}
In\label{yearoptional} many cases the correct citation of a case depends on whether the
year forms part of the citation (as it does in [1996] AC 155), or
merely provides additional information (as it does in (2001) 49
BMLR 1). In general, the \oscola\ style proceeds as follows (in
relation to cases and journal articles):
\begin{enumerate}
\item If no \texttt{volume} field is present, it \emph{assumes} that
  the year is an essential part of the citation, and prints it in
  brackets,\footnote{Except in Scottish cases, where the convention is for it to be printed without any adornment.} as [\textsc{year}].
\item If a \texttt{volume} field is present, it \emph{assumes}
  that the year is not an essential part of the citation and prints it
  in parentheses, as (\textsc{year}).
\end{enumerate}

\index[general]{options field@\texttt{options} field!year-essential@\texttt{year-essential}}You can override this choice by specifying the \texttt{option},
\texttt{year-essential=true}. You will always need to do this if the
report you are citing has both a year and a volume number, but the
year is an essential part of the citation, which should be printed in square brackets.

\subsection{Countries\label{countries}}

\index[general]{keywords field@\texttt{keywords} field!countries}
\index[general]{bibfile@\texttt{.bib} file!keywords field@\texttt{keywords} field!countries}
\index[general]{countries!keywords@\texttt{keywords}}
\index[general]{jurisdictions|see {countries}}
The \texttt{keywords} field should be set to the jurisdiction from
which the case comes. The jurisdictions that \oscola\ recognises are
given in table \ref{juristable}. If no jurisdiction is specified, the
case will be assumed to be English.\footnote{It is, therefore,
  unnecessary to specify a jurisdiction for English cases or statutes;
  but it is good practice to do so. In particular, while it is
  acceptable to leave the keywords field blank, if there is
  \emph{any} keyword, then you need to make sure there is also a
  jurisdiction label, or the indexing will not work.}

The identification of the relevant jurisdiction is sometimes important
for formatting (\textsc{eu}, \textsc{echr} and Scottish cases are
formatted differently from English ones, as are American cases), but
it is always important if you are going to be using \oscola\ to create
complex tables of cases, so try to make sure it is right.

\begin{table}
\centering
\index[general]{keywords field@\texttt{keywords} field!countries, list of}
\index[general]{countries!list of}
\begin{tabular*}{10cm}{l p{8cm}}
\toprule
\texttt{gb}   & United Kingdom \\
\texttt{en}   & England \\
\texttt{cy}   & Wales \\
\texttt{sc}   & Scotland \\
\texttt{ni}   & Northern Ireland \\
\texttt{eu}   & European Union (including the \textsc{eec}, \textsc{ec},
	        \textsc{ecsc} and \textsc{euratom}) \\
\texttt{echr} & Organs of the Council of Europe dealing with the
	        European Convention on Human Rights \\
\texttt{int}  & (Public) international law cases and materials \\
\texttt{us}   & United States \\
\texttt{ca}   & Canada \\
\texttt{aus}  & Australia \\
\texttt{nz}   & New Zealand \\\bottomrule
\end{tabular*}
\caption{Jurisdiction Abbreviations\label{juristable}}
\end{table}

In some cases (notably American, Australian and Canadian states or
provinces) it may also be necessary to set the \texttt{location}
field, in order to ensure that citations are properly given.
\index[general]{cases!in general|)}

\subsection{Indexing}

\index[general]{indexing!cases}
\index[general]{cases!indexing}

With the sole exception of \textsc{eu} cases, which have slightly more
complex indexing methods, all cases are `sent' to an index ordered
alphabetically by \texttt{indexsorttitle}, \texttt{indextitle} or
\texttt{title}.

\section{English Cases}

\index[general]{cases!English}
English cases should be entered in the database using the fields
explained above. The keywords field should be set to
\texttt{en}, except in the case of House of Lords cases for which it should
technically be set to \texttt{gb}, though in practice it will do no
harm if it is set to \texttt{en}. Indeed, since \oscola\ will assume
that any case is English unless told different, you can if you like
omit it altogether.

\subsection{Some Basic Examples}

\index[general]{cases!English!examples|(}
The following two examples,\footcite[13]{oscola} show very basic
usage.
\begin{bibexample}[corr08]
\begin{verbatim}
@jurisdiction{corr08,
  title        = {Corr v. I.B.C. Vehicles Ltd.},
  keywords     = {gb},
  date         = {2008},
  number       = {[2008] UKHL 13},
  journaltitle = {A.C.},
  volume       = {1},
  pages        = {884},
  options      = {year-essential=true},
  institution  = {HL},
  shorttitle   = {Corr},
  pagination   = {[]},
}
\end{verbatim}
\end{bibexample}

\begin{bibexample}[page96]
\begin{verbatim}
@jurisdiction{page96,
 title         = {Page v Smith},
 usera         = {gb},
 date          = {1996},
 journaltitle  = {AC},
 pages         = {155},
 institution   = {HL},
}
\end{verbatim}
\end{bibexample}

\begin{description}
\item[\egcite{[11]\{corr08\}}] \cite[11]{corr08}
\item[\egcite{\{page96\}}] \cite{page96}
\end{description}

Note: (1) it was not necessary to specify that the year was essential
in example \ref{page96} because there is no volume number, so it can
be assumed to be essential, but it was necessary to do so in example
\ref{corr08}, because there is a volume number there, so \oscola\
cannot infer that the year is essential. (2) As explained above,
although there was information about the institution included in the
bibliography data for example \ref{corr08}, \oscola\ knew that it
could be ignored. It also correctly stripped out full points. (3) The
\texttt{pagination} field of example \ref{corr08} ensures that
pinpoints are correctly formatted.

Now let's look at a case where the year is not
essential.\footcite[14]{oscola}

\begin{bibexample}[barrett01a]
\begin{verbatim}
@jurisdiction{barrett01a,
  title        = {Barrett v Enfield LBC},
  date         = {2001},
  journaltitle = {BMLR},
  volume       = {49},
  pages        = {1},
  institution  = {HL},
}
\end{verbatim}
\end{bibexample}
\verb|\cite{barrett01a}| produces \cite{barrett01a}. As you can see,
since there is a volume number given, the package `knows' that the
year must be a non-essential part of the citation, and formats
accordingly.
\index[general]{cases!English!examples|)}

\subsection{Unreported Cases}

\index[general]{cases!English!unreported}
If you have a case that is unreported, just leave the
\texttt{reporter} field empty. How these cases will be formatted
will depend on whether there is a neutral citation in the
\texttt{number} field. (If there isn't a neutral
citation, \oscolashort\ requires you to give, if you can, a
precise date, not just a year: see example \ref{stubbs90}.)

\begin{bibexample}[stubbs90]
\begin{verbatim}
@jurisdiction{stubbs90,
  title        = {Stubbs v Sayer},
  institution  = {CA},
  date         = {1990-11-08},
}
\end{verbatim}
\end{bibexample}

\begin{bibexample}[calvert02]
\begin{verbatim}
@jurisdiction{calvert02,
  title        = {Calvert v Gardiner},
  number       = {[2002] EWHC 1394 (QB)},
  institution  = {QB},
  date         = {2002-01-01},
}
\end{verbatim}
\end{bibexample}

\begin{description}
\item[\egcite{\{stubbs90\}}] \cite{stubbs90}
\item[\egcite{\{calvert02\}}] \cite{calvert02}
\end{description}

You will notice that in this case I didn't bother to give a
\texttt{keywords} field---so \texttt{en} is assumed.

\subsection{Newspaper Reports}

\index[general]{cases!English!newspaper reports}
\index[general]{newspapers!case reports}
Reports in newspapers require only the \texttt{title},
\texttt{journaltitle}, \texttt{date} (which should be the date the
report was published, in full) and \texttt{institution}. To make sure
the formatting is as it should be, you need to enter an \texttt{entrysubtype} of \texttt{newspaper}.

\begin{bibexample}[powick93]
\begin{verbatim}
@jurisdiction{powick93,
  title        = {Powick v Malvern Wells Water Co},
  date         = {1993-09-28},
  journaltitle = {The Times},
  institution  = {QB},
  entrysubtype = {newspaper},
}
\end{verbatim}
\end{bibexample}

\begin{description}
\item[\egcite{\{powick93\}}] \cite{powick93}
\end{description}
\noindent
(This is correct,\footcite[See][18]{oscola} though I
think it looks wrong. In other cases, OSCOLA requires that one put a
comma between different parts of a citation which are in text, even if
there are different fonts. I can't see why it doesn't
here:\footnote{In other forms of case citation a comma is unnecessary,
  because the (year) or [year] marks a clear division.} but it
doesn't.)

\subsection{Old cases reprinted in the English Reports\label{parallel:reports}}

\index[general]{English Reports|(}
\index[general]{cases!English!pre-1865|(}

Where a case appears in one of the old nominate reports and is also
reprinted in the English Reports or the Revised Reports, \oscolashort\
requires that you give references to both the original nominate report
and to the English Reports.\footcite[20--21]{oscola} So in such cases
you need to define two reports, and to use two pinpoints.

Complete all the usual fields for the report reference
(\texttt{reporter}, \texttt{volume}, \texttt{series} and so forth)
with the details of the original nominate report. Enter the `parallel'
citation to the English Reports as follows: \texttt{parvolume} holds the
report volume number; \texttt{parreporter} holds the report title;
\texttt{parseries} holds the series (if any) and \texttt{parpages} holds the
page reference.

If you simply want to cite the first page of the case, without any
pinpointing, then use the \verb|\cite{}| or \verb|\footcite{}| command
as usual, and the reference will be formatted as per \oscola, with two
references separated by commas.

If you want to give pinpoint citations, you need to specify them in
the \texttt{postnote} argument by separating them with the pipe character: \texttt{|}.\index[general]{parallel citations}

\begin{bibexample}[henly28]
\begin{verbatim}
@jurisdiction{henly28,
  title      = {Henly v Mayor of Lyme},
  date       = {1828},
  volume     = {5},
  reporter   = {Bing},
  pages      = {91},
  keywords   = {en},
  parvolume  = {130},
  parreporter= {ER},
  parpages   = {995},
}
\end{verbatim}
\end{bibexample}

\begin{description}%[labelwidth=5cm,labelsep=0.3cm,leftmargin=5.3cm]
\item[\egcite{\{henly28\}}] \fullcite{henly28}
\item[\egcite{[93|995]\{henly28\}}] \cite[93|995]{henly28}
\end{description}

Cases with parallel citations will never be given in shortened form,
if there is a pinpoint citation, because the resulting citation (`(n
\emph{x}) 23; 40') would be confusing. Nor will `ibid' be used, for
the same reason

\index[general]{English Reports|)}
\index[general]{cases!English!pre-1865|)}

\section{Scottish Cases}

\index[general]{cases!Scottish|(}
For Scottish cases, follow the same pattern and basic fields as you
use for English cases, but setting \texttt{keywords} to \texttt{sc}.

Scottish cases will be formatted in the same way as English
cases,\footcite[See][22--23]{oscola} except that where the year is an
essential part of the citation it is set \emph{without brackets}
rather than with square brackets in the English style (but neutral
citations do have square brackets).

Decisions of the House of Lords and the Supreme Court should have
\texttt{keywords} set to \texttt{gb} (not \texttt{sc}), in order to
ensure correct indexing. This means, however, that if you cite to a
scottish report in which square brackets should not be used,
\oscola\ needs help (or it will assume that it should follow the
English style for such cases). In such a case use the option
\texttt{scottish-style = true}. Example \ref{davidson05} shows this.\index[general]{options field@\texttt{options} field!scottish-style@\texttt{scottish-style}}

\begin{bibexample}[hislop42]
\begin{verbatim}
@jurisdiction{hislop42,
  title        = {Hislop v Durham},
  date         = {1842},
  volume       = {4},
  reporter     = {D},
  pages        = {1168},
  keywords     = {sc},
}
\end{verbatim}
\end{bibexample}

\begin{bibexample}[adams03]
\begin{verbatim}
@jurisdiction{adams03,
  title        = {Adams v Advocate General},
  date         = {2003},
  reporter     = {SC},
  pages        = {171},
  institution  = {OH},
  keywords     = {sc},
}
\end{verbatim}
\end{bibexample}

\begin{bibexample}[dodds03]
\begin{verbatim}
@jurisdiction{dodds03,
  title        = {Dodds v HM Advocate},
  date         = {2003},
  reporter     = {JC},
  pages        = {8},
  keywords     = {sc},
}
\end{verbatim}
\end{bibexample}

Example \ref{crofters02} shows a case report in the `Land Ct' series
of the SLT. I've chosen to put this in the \texttt{series}
field---though frankly it would make no difference to include it as
part of the \texttt{reporter}.

\begin{bibexample}[crofters02]
\begin{verbatim}
@jurisdiction{crofters02,
  title        = {Crofters Commission v. Scottish Ministers},
  date         = {2002},
  reporter     = {SLT},
  series       = {(Land Ct)},
  pages        = {19},
  keywords     = {sc},
}
\end{verbatim}
\end{bibexample}

Note, in example \ref{davidson05} that the \texttt{option} of
\texttt{scottish-style} is set in order to prevent \oscola\ using the
default English style for a UK case.

\begin{bibexample}[davidson05]
\begin{verbatim}
@jurisdiction{davidson05,
  title        = {Davidson v. Scottish Ministers},
  date         = {2006},
  number       = {[2005] UKHL 74},
  reporter     = {SC (HL)},
  keywords     = {gb},
  options      = {scottish-style=true},
}
\end{verbatim}
\end{bibexample}

\begin{bibexample}[smart06]
\begin{verbatim}
@jurisdiction{smart06,
  title        = {Smart v HM Advocate},
  number       = {[2006] HJAC 12},
  date         = {2006},
  journaltitle = {JC},
  pages        = {119},
  keywords     = {sc},
}
\end{verbatim}
\end{bibexample}

\begin{description}
\item[\egcite{\{hislop42\}}] \cite{hislop42}
\item[\egcite{\{adams03\}}]  \cite{adams03}
\item[\egcite{\{dodds03\}}]  \cite{dodds03}
\item[\egcite{[25]\{crofters02\}}] \cite[25]{crofters02}
\item[\egcite{\{davidson05\}}] \cite{davidson05}
\item[\egcite{\{smart06\}}] \cite{smart06}
\end{description}

\index[general]{cases!Scottish|)}

\section{European Union Cases}
\index[general]{cases!European Union|(} 


\subsection{Decisions of the ECJ and the General Court}
\index[general]{cases!European Union!courts|(}
Decisions of the ECJ and the General Court should be entered in the
database using the \texttt{@jurisdiction} type. Use all the usual
fields described in relation to English cases, but
\begin{description}
\item[keywords] Must include \texttt{eu}.
\item[number]
  \index[general]{number field@\texttt{number} field!EU cases@in EU cases}
  \index[general]{bibfile@\texttt{.bib} file!number field@\texttt{number field}}
  \index[general]{cases!European Union!case number@\texttt{case number}}
  Enter the case number (or numbers) in the
  \texttt{number} field. Do not include `Case' or `Joined Cases' in the number field.
\item[type] If your case is an ordinary one (ie, `Case' or `Joined
  Cases') you need enter nothing here. If you need to specify
  something unusual (like \texttt{Opinion}) to precede the case
  number, use the \texttt{type} field.\index[general]{type field@\texttt{type} field}\index[general]{bibfile@\texttt{.bib} file!type field@\texttt{type} field}
\item[options] It is not necessary to specify
  \texttt{year-essential=true} for decision in the European Court
  Reports (ECR) -- though it will do no harm to do so.
\item[reporter] If the case is not yet reported in the ECR (or
  some other full report) you can include a citation to the Official
  Journal. Follow the advice set out in relation to Commission
  decisions, below.
\item[pagination] You can, if you like, specify \texttt{paragraph};
  but if nothing is entered, paragraph will be assumed.
\end{description}

\index[general]{cases!European Union!case number}
Because, unless \texttt{type} is set, \oscola\ needs to analyse the
case number in order to decide whether to print `Case' or `Joined
Cases', the format of the number is important. If the number contains
a comma, an en-dash (\verb|--|), or the word `and', then the package
knows that there is more than one case. However, this does mean that
you can't use the en-dash in the case number, because if the package
sees \verb|C--2/98| it will think it is looking at more than one
case. There are two things you could do. Either use a single hypen in
case numbers (\texttt{C-2/98}) or use a special command\index[general]{textendash@\texttt{\textbackslash textendash}, in EU cases}
\verb|\textendash|, which will print an en-dash but will not lead
the package to make mistakes about the number of cases:
\verb|C\textendash/98|. As to the choice, \oscolashort\ itself seems to use
an en-dash, which looks typographically best to me; but the ECJ and
Community reports themselves use a hyphen, so it could certainly be
argued that a hyphen is correct, and it's simpler. Obviously, be
consistent.

\index[general]{cases!European Union!examples|(}
\subsubsection{Examples}
Examples \ref{abdhu} to \ref{C556/07} show the basic format for a
variety of simple reported cases.

\begin{bibexample}[abdhu]
\begin{verbatim}
@jurisdiction{adbhu,
  keywords        = {eu},
  title           = {Procureur de la R{\'e}publique v ADBHU},
  shorttitle      = {ADBHU},
  number          = {240/83},
  institution     = {ECJ},
  date            = {1985},
  journaltitle    = {ECR},
  pages           = {531},
}
\end{verbatim}
\end{bibexample}

Example \ref{C430/93} shows how it can sometimes be a good idea to
include \texttt{hyphenation} information for \textsc{eu} cases, which
can help to ensure proper linebreaking.
\begin{bibexample}[C430/93]
\begin{verbatim}
@jurisdiction{C430/93,
  keywords       = {eu},
  title          = {Jereon van Schijndel v Stichting 
                    Pensioenfonds voor Fysiotherapeuten},
  shorttitle     = {van Schijndel},
  number         = {C-430/93--C-431/93},
  institution    = {ECJ},
  journaltitle   = {ECR},
  volume         = {I},
  pages          = {4705},
  hyphenation    = {dutch},
}
\end{verbatim}
\end{bibexample}

\begin{bibexample}[Ex3]
\begin{verbatim}
@jurisdiction{T344/99,
  keywords      = {eu},
  title         = {Arne Mathisen AS v Council},
  shorttitle    = {Arne Mathisen},
  number        = {T-344/99},
  institution   = {CFI},
  journaltitle  = {ECR},
  volume        = {II},
  pages         = {2905},
  date          = {2002},
}
\end{verbatim}
\end{bibexample}

\index[general]{cases!European Union!unreported}
Example \ref{C556/07} shows a case that is not (yet) reported in the
ECR reports, but available only in the Official Journal. Although
\oscolashort\ suggests that in those circumstances the citation should
be to OJ, my own view would be that in such cases it is probably
better either to give an ECR reference with the pages set to 0000---as
is very common practice---leaving it to the reader to find the case
online, which is not difficult, or to give an online reference.

\begin{bibexample}[C556/07]
\begin{verbatim}
@jurisdiction{C556/07,
  keywords     = {eu},
  title        = {Commission v France},
  date         = {2009},
  journaltitle = {OJ},
  series       = {C},
  volume       = {102},
  pages        = {8},
  number       = {C-556/07},
  institution  = {ECJ},
}
\end{verbatim}
\end{bibexample}

\begin{description}[leftmargin=3.6cm,labelwidth=3.4cm,labelsep=0.2cm]
\item[\egcite{\{adbhu\}}]\cite{adbhu}
{\sloppy\item[\egcite{\{C430/93\}}]\cite{C430/93}}
\item[\egcite{\{T344/99\}}]\cite{T344/99}
\item[\egcite{\{C556/07\}}]\cite{C556/07}
\end{description}

\index[general]{cases!European Union!examples|)}

Note, in example \ref{C176/03}, that even if pagination is not
defined, citations are to paragraphs as \oscolashort\footcite[30]{oscola}
requires. (This also illustrates the use of \verb|\textendash| in the
case number.)

\begin{bibexample}[C176/03]
\begin{verbatim}
@jurisdiction{C176/03,
  keywords     = {eu},
  title        = {Commission v Council},
  number       = {C\textendash176/03},
  date         = {2005},
  journaltitle = {ECR},
  volume       = {I},
  pages        = {7879},
  institution  = {ECJ},
}
\end{verbatim}
\end{bibexample}

\begin{description}
\item[\egcite{[47--48]\{C176/03\}}]\cite[47--48]{C176/03} 
\end{description}

\subsubsection{Indexing}

\index[general]{cases!European Union!indexing}
\index[general]{indexing!EU cases}
Each citation of an \textsc{eu} case will produce an entry in two separate `virtual' indices: the \texttt{eucases} index, which is alphabetical, and produces citations in the form:
\begin{center}
\citeinindex{C176/03}
\end{center}
and the \texttt{eucasesnum} index, which orders cases numerically, and produces citations in the form
\begin{center}
\citeinindexnum{C176/03}
\end{center}
Obviously you will not want to mix these indexes together. The alphabetical type alone is appropriate for `mixing' with other case tables.

\index[general]{cases!European Union!courts|)}

\subsection{Commission Decisions}
\index[general]{cases!European Union!Commission decisions|(}

\subsubsection{Basic usage}

Commission Decisions\label{eudecisioncases} in competition and merger
matters are, according to \oscolashort,\footcite[30--31]{oscola} cited
as if they were cases. So, rather than treating them as legislation,
give them the entry type of \texttt{@jurisdiction} type.

The requirements here are a bit intricate:
\begin{itemize}
\item Such decisions have \emph{two} numbers: the `case number' and
  the formal number of the decision itself. Both are required. Put the
  formal decision number in the \texttt{number} field, and the
  Commission case number in the \texttt{userb} field.\index[general]{userb field@\texttt{userb} field}
\index[general]{bibfile@\texttt{.bib} file!userb field@\texttt{userb} field}
\item The reference will usually be to the Official Journal (OJ). For
  such references, you need to set \texttt{journaltitle} as
  \texttt{OJ}, and then specify the series (as \texttt{L} or
  \texttt{C}) in the \texttt{series} field, the `issue' in the
  \texttt{volume} field or the \texttt{issue} field, and then give the
  page number.\index[general]{Official Journal!refences to}
\end{itemize}

\begin{bibexample}[alcatel]
\begin{verbatim}
@jurisdiction{alcatel,
  keywords     = {eu},
  title        = {Alcatel/Telettra},
  userb        = {IV/M.042},
  number       = {91\slash 251\slash EEC},
  institution  = {Commission},
  date         = {1991},
  journaltitle = {OJ},
  series       = {L},
  volume       = {122},
  pages        = {48},
}
\end{verbatim}
\end{bibexample}

\begin{bibexample}[verkehrsorgani]
\begin{verbatim}
@jurisdiction{verkehrsorgani,
  keywords     = {eu},
  title        = {Georg Verkehrsorgani v Ferrovie 
                  dello Stato},
  userb        = {Case COMP\slash 37.685},
  number       = {2004\slash 33\slash EC},
  institution  = {Commission},
  year         = {2004},
  journaltitle = {OJ},
  series       = {L},
  issue        = {11},
  pages        = {17},
}
\end{verbatim}
\end{bibexample}

\begin{description}
\item[\egcite{\{alcatel\}}] \cite{alcatel} \\
\item[\egcite{\{verkehrsorgani\}}] \cite{verkehrsorgani}
\end{description}


\index[general]{cases!European Union!Commission decisions|)}

\subsubsection{Indexing}
\index[general]{indexing!EU Commission decisions}

Commission decision which have been given the \texttt{@jurisdiction}
entry type are indexed in \texttt{eucases}, alphabetically by
title. They are \emph{not} indexed in the \texttt{eucasesnum} index
because the format of the numbers is so different from that of other
cases that it would only cause confusion. At present, there is no
facility to produce a numerically ordered table of Commission
decisions. This is, I am afraid, a missing feature. It might be added
to future updates of \oscola, if it seems to be required: but it is
not straightforward.

\index[general]{cases!European Union|)} 

\section{European Human Rights Cases}

\subsection{European Court of Human Rights}
\index[general]{cases!European Court of Human Rights|(}
\index[general]{European Convention on Human Rights!cases|(}

\textsc{Oscola} recommends\footcite[31]{oscola} that cases decided by the
European Court of Human Rights should be cited to either the official
reports (Series A, until 1998, and thereafter the ECHR), or to the
unofficial \emph{European Human Rights Reports}. Because the
formatting requirements for Series A and ECHR are rather unusual, the
package detects such references and formats accordingly. EHRR
citations follow the usual `English' pattern and are handled
accordingly. The \texttt{keywords} field should be set to \texttt{echr}.

Apart from that, the fields are the same as for English cases. If
citing Series A or the EHRR put the case identification number in the
\texttt{pages} field. (This is different from the \emph{application
  number}, which should be put in the \texttt{number} field---though
\oscola\ does not use it if the case is reported. Pagination will be
set to \texttt{paragraph} (so that pinpoint citations are to `para' or
`paras', as OSCOLA requires) unless you specify something different.

\begin{bibexample}[johnson86]
\begin{verbatim}
@jurisdiction{johnston86,
  title        = {Johnston v Ireland},
  reporter     = {Series A},
  pages        = {122},
  date         = {1986},
  institution  = {ECtHR},
  keywords     = {echr},
}
\end{verbatim}
\end{bibexample}

\begin{bibexample}[osman98]
\begin{verbatim}
@jurisdiction{osman98,
  title        = {Osman v UK},
  reporter     = {ECHR},
  date         = {1998},
  volume       = {8},
  pages        = {3124},
  institution  = {ECtHR},
  keywords     = {echr},
}
\end{verbatim}
\end{bibexample}

\begin{bibexample}[omojudi10]
\begin{verbatim}
@jurisdiction{omojudi10,
  title        = {Omojudi v UK},
  reporter     = {EHRR},
  date         = {2010},
  pages        = {10},
  volume       = {51},
  institution  = {ECtHR},
  keywords     = {echr},
}
\end{verbatim}
\end{bibexample}

\begin{description}
\item[\egcite{\{johnston86\}}] \cite{johnston86}
\item[\egcite{\{osman98\}}] \cite{osman98} 
\item[\egcite{[10]\{omojudi10\}}] \cite[10]{omojudi10}
\end{description}

\index[general]{cases!European Court of Human Rights!unreported}
If a case is unreported, leave the \verb|reporter| blank in the usual
way. \textsc{Oscola} requires you to provide an application number in such cases which
should be entered in the \texttt{number} field \emph{without} `App no', which
will be added automatically. You should also provide a \emph{full}
date, and make sure that you have completed the \texttt{institution}
field. (When the case is subsequently reported you can simply add the
report details, without deleting the application number or changing
the date: \oscola\ will ignore the information to the extent that it
is no longer required.)

\begin{bibexample}[balogh04]
\begin{verbatim}
@jurisdiction{balogh04,
  title        = {Balogh v Hungary},
  number       = {47940/99},
  date         = {2004-07-20},
  institution  = {ECtHR},
  keywords     = {echr},
}
\end{verbatim}
\end{bibexample}

\begin{description}
\item[\egcite{\{balogh04\}}] \cite{balogh04}
\end{description}

\index[general]{cases!European Court of Human Rights|)}

\subsection{Commission Decisions}
\index[general]{cases!European Commission on Human Rights|(}

Simply follow the instructions for cases decided by the European Court
of Human Rights, but specify the institution as
\texttt{Commission}. Whether this will be printed depends on whe\-ther
the report you have cited is one which is devoted to Commission
Decisions---the \emph{Decisions and Reports} (DR) and \emph{Collection
  of Decisions} (CD)---in which case it will be silently ignored, or
some other series of reports, in which case `Commission Decision'
will, as OSCOLA requires, be printed at the end of the citation.

\begin{bibexample}[x71]
\begin{verbatim}
@jurisdiction{x71,
  title        = {X v Netherlands},
  date         = {1971},
  volume       = {38},
  journaltitle = {CD},
  pages        = {9},
  institution  = {Commission},
  keywords     = {echr},
}
\end{verbatim}
\end{bibexample}

\begin{bibexample}[ccsu87]
\begin{verbatim}
@jurisdiction{ccsu87,
  title        = {Council of Civil Service Unions v UK},
  date         = {1987},
  volume       = {10},
  journaltitle = {EHRR},
  pages        = {269},
  institution  = {Commission},
  keywords     = {echr},
}
\end{verbatim}
\end{bibexample}

\begin{bibexample}[simpson86]
\begin{verbatim}
@jurisdiction{simpson86,
  title        = {Simpson v UK},
  date         = {1989},
  volume       = {64},
  journaltitle = {DR},
  pages        = {188},
  institution  = {Commission},
  keywords     = {echr},
}
\end{verbatim}
\end{bibexample}

\begin{description}
\item[\egcite{\{x71\}}] \cite{x71} 
\item[\egcite{\{ccsu87\}}] \cite{ccsu87}
\item[\egcite{\{simpson86\}}] \cite{simpson86}
\end{description}

\index[general]{cases!European Commission on Human Rights|)}
\index[general]{European Convention on Human Rights!cases|)}

\section{Canadian Cases}

\index[general]{cases!Canadian}
For Canadian cases, use the same fields as you use for English
cases. The only `trick' is that you will need to be discerning about
giving court and location information. If provided these will be
printed, unless there is a neutral citation (in which case it is
assumed that the neutral citation information provides the necessary
information about court and location). The citations will essentially
be formatted in accordance with the McGill
guide,\footcite{mcgill:guide} except that whereas the McGill guide
prints the pinpoint after the neutral citation and before the report
citation, \oscola\ will follow the pattern it uses for English cases,
where the citation -- which, where there is a neutral citation, should
be to a paragraph -- will be printed after the report information;
there may also be fewer commas. So whereas McGill would
have \begin{center} \emph{Clearbrook Ironworks Ltd v Letorneau}, 2006
  FCA 42 \textbf{at para 3}, 46 CPR (4th) 241
\end{center} \oscola\ will produce \begin{center} \cite{clearbrook}, \textbf{para 3}
\end{center}

\begin{bibexample}[clearbrook]
\begin{verbatim}
@jurisdiction{clearbrook,
  title          = {Clearbrook Ironworks Ltd. v. Letorneau},
  number         = {2006 FCA 42},
  volume         = {46},
  reporter       = {C.P.R.},
  series         = {4th},
  pages          = {241},
  date           = {2006},
  pagination     = {paragraph},
  keywords       = {ca},
}
\end{verbatim}
\end{bibexample}

\section{Australian Cases}

\index[general]{cases!Australian|(}
Follow the same format as is used for English cases. You may use the \verb|location| field to identify the jurisdiction
from which the case comes. Note that neither court nor location
field should be used if the origin of the case is obvious from the
citation.\footnote{These examples are drawn from \cite[37]{melbourne}.}

\begin{bibexample}[tang]
\begin{verbatim}
@jurisdiction{tang,
  title          = {R. v. Tang},
  date           = {2008},
  volume         = {237},
  reporter       = {C.L.R.},
  pages          = {1},
  keywords       = {au},
}
\end{verbatim}
\end{bibexample}

\begin{bibexample}[bakker]
\begin{verbatim}
@jurisdiction{bakker,
  title          = {Bakker v. Stewart},
  date           = {1980},
  reporter       = {V.R.},
  pages          = {17},
  keywords       = {au},
}
\end{verbatim}
\end{bibexample}

\begin{description}
\item[\egcite{[7]\{tang\}}] \cite[7]{tang}
\item[\egcite{[22]\{bakker\}}] \cite[22]{bakker}
\end{description}
\index[general]{cases!Australian|)}

\section{New Zealand Cases}

\index[general]{cases!New Zealand}
Follow the same format as is used for English cases.

\section{US Cases}

\index[general]{cases!United States}
Use the same fields as are used for English cases. Be careful about
specifying courts: apart from the US Supreme Court (where, if you cite
the US Reports, no court identifier will ever be printed), the court
and location fields will be used if present.

\index[general]{cases!United States!docket number}
\index[general]{number field@\texttt{number} field!US cases@in US cases}
\index[general]{bibfile@\texttt{.bib} file!number field@\texttt{number} field}
For unreported cases, you may specify a \verb|number| (the docket
number) and optionally something in the \verb|eprint| field, which can
\index[general]{cases!United States!lexis number@\textsc{lexis} number}
\index[general]{cases!United States!westlaw number@WestLaw number}
\index[general]{eprint field@\texttt{eprint} field}
\index[general]{bibfile@\texttt{.bib} file!eprint field@\texttt{eprint} field}
be used for a \textsc{Lexis} or WestLaw number: see example
\ref{operadora09}.

Although you could use the \verb|series| field for reports (as in
example \ref{western97})\footnote{In which case it will be printed in
  US style: 1st, 2d, 3d.} in my view it is often simpler just to make
the series part of the \verb|reporter| field, as in example
\ref{harbor91}. You might note also, there, that if you enter fields
using full points in the standard American way, the spacing is taken
care of automatically.

\begin{bibexample}[harbor91]
\begin{verbatim}
@jurisdiction{harbor91,
  title        = {Harbor Insurance Co. v. 
                  Tishman Construction Co.},
  volume       = {578},
  reporter     = {N.E.2d.},
  pages        = {1197},
  location     = {Ill.},
  court        = {App. Ct.},
  date         = {1991},
  keywords     = {us},
}
\end{verbatim}
\end{bibexample}

\begin{bibexample}[western97]
\begin{verbatim}
@jurisdiction{western97,
  title        = {Western Alliance Insurance Co. v. 
                  Gill},
  volume       = {686},
  reporter     = {N.E.},
  series       = {2},
  pages        = {997},
  location     = {Mass.},
  date         = {1997},
  keywords     = {us},
}
\end{verbatim}
\end{bibexample}

\begin{bibexample}[operadora09]
\begin{verbatim}
@jurisdiction{operadora09,
  title        = {Re Operadora D.B. Mexico S.A.},
  indextitle   = {Operadora D.B. Mexico S.A., Re},
  number       = {6:09-cv-383-Orl022GJK},
  eprint       = {2009 WL 2423138},
  court        = {M.D. Fla.},
  date         = {2009},
  keywords     = {us},
}
\end{verbatim}
\end{bibexample}

\begin{description}
\item[\egcite{[1199]\{harbor91\}}] \cite[1199]{harbor91}
\item[\egcite{\{western97\}}] \cite{western97}
\item[\egcite{\{operadora\}}] {\sloppy\cite{operadora09}}
\end{description}
\index[general]{cases!United States|)}

\section{Public International Cases}

As explained in section \ref{scope}, \oscola\ does not provide
fully comprehensive provisions for citing all possible forms of
international cases. However, it does include provision which should
be adequate for the decisions of the ICJ and for at least a reasonable
number of other public international law cases.

\subsection{Fields and usage}

\index[general]{cases!International|(}
\index[general]{Public International Law|see{International Law}}
\index[general]{International Law!cases|(}
The fields will mostly be familiar. You should set \texttt{keywords}
to include \texttt{int} to show an international case. The most
distinctive feature are the additional parts of the case title. The
parties to the case should be given in the \texttt{subtitle}
field.
\index[general]{subtitle field@\texttt{subtitle} field}
\index[general]{bibfile@\texttt{.bib} file!subtitle field@\texttt{subtitle} field}
International cases often require additional explanation about
the nature of the decision (such as whether it relates to jurisdiction
or the merits): add this in the \texttt{titleaddon} field. Both
\texttt{subtitle} and \texttt{titleaddon} get printed in brackets
after the main citation: but whereas the \texttt{subtitle} field is
italic, the \texttt{titleaddon} field is in roman type.
\index[general]{titleaddon field@\texttt{titleaddon} field}
\index[general]{bibfile@\texttt{.bib} file!titleaddon field@\texttt{titleaddon} field}
\index[general]{cases!International!subtitle field@use of \texttt{subtitle} field}
\index[general]{cases!International!titleaddon field@use of \texttt{titleaddon} field}

\index[general]{cases!International!PCIJ Reports}
\index[general]{PCIJ Reports}
In the case of decision in the PCIJ Reports, you need not specify
\texttt{Series A} or \texttt{Series A/B} in the \texttt{series} field,
just \texttt{A} or \texttt{A/B}. The number should go in the
\texttt{pages} field, since that is the purpose it really serves: but
you need not specify \texttt{No}---it will be added automatically.

\index[general]{cases!International!fields}
\begin{description}
\item[keywords] Include \texttt{int}.
\item[title] As in other cases.
\item[subtitle] Give the names of the parties.
\item[subtitleaddon] Give any additional information about the
  particular aspect of the case that is being reported, such as
  `merits' or `jurisdiction'.
\item[volume etc] As in other
  \texttt{@jurisdiction} types. Note the comments above concerning the
  reports of the PCIJ.
\item[institution] Can be used for institutions other than the
  ICJ. For the ICJ/PCIJ, the field will not be used so long as the report
  cited is the ICJ reports or the PCIJ reports.
\end{description}

\begin{bibexample}[corfu]
\begin{verbatim}
@jurisdiction{corfu,
 title         = {Corfu Channel Case},
 subtitle      = {UK v Albania},
 titleaddon    = {Merits},
 date          = {1949},
 journaltitle  = {ICJ Rep},
 pages         = {4},
 keywords      = {int},
}
\end{verbatim}
\end{bibexample}

\begin{bibexample}[nicaragua]
\begin{verbatim}
@jurisdiction{nicaragua,
  title        = {Land, Island and Maritime Frontier
                  Case},
  subtitle     = {El Salvador\slash Hon\-duras, Nicaragua
                  intervening},
  titleaddon   = {Application for Intervention},
  date         = {1994},
  journaltitle = {ICJ Rep},
  pages        = {92},
  keywords     = {int},
}
\end{verbatim}
\end{bibexample}

\begin{bibexample}[wall]
\begin{verbatim}
@jurisdiction{wall,
  title        = {Legal Consequences of the Construction
                  of a Wall},
  titleaddon   = {Advisory Opinion},
  date         = {2004},
  url          = {http://www.icj-cij.org/icjwww/imwp/imwpframe.htm},
  urldate      = {2005-07-21},
  pagination   = {[]},
  keywords     = {int},
}
\end{verbatim}
\end{bibexample}

\begin{bibexample}[chorzow]
\begin{verbatim}
@jurisdiction{chorzow,
  title        = {Case Concerning the Factory at Chorz{\'o}w},
  indextitle   = {Chorz{\'o}w Factory Case},
  subtitle     = {Germany v Poland},
  titleaddon   = {Merits},
  journaltitle = {PCIJ Rep},
  series       = {A},
  pages        = {17},
  keywords     = {int},
  date         = {1929},
}
\end{verbatim}
\end{bibexample}

\begin{description}
\item[\egcite{\{corfu\}}] \cite{corfu}
\item[\egcite{\{nicaragua\}}] \cite{nicaragua}
\item[\egcite{\{wall\}}] \cite{wall}\footnote{This is the \textsc{url} printed in \oscolashort. It is not correct!}
\item[\egcite{\{chorzow\}}] \cite{chorzow}
\end{description}

\subsection{Indexing}

\index[general]{cases!International!indexing}
\index[general]{indexing!International Law cases}
International cases are indexed alphabetically by title in the
\texttt{pilcases} virtual index. No attempt is made to further
sub-divide them, and in a complex work you might well want to make use
of the \texttt{tabulate} field to do so.

\section{Legislation}

\index[general]{legislation|(}
The \oscola\ style provides support for citing as `legislation' (a) UK
Acts of Parliament (and those of the Scottish and Welsh Assemblies and
the Northern Ireland Parliament), (b) secondary legislation (statutory
instruments) made in any part of the UK, (c) draft UK primary
legislation (bills), (d) the treaties establishing
the \textsc{eu}, (e) legislation made by the \textsc{eu}. It does not
provide facilities for citing non-UK or \textsc{eu}
legislation. Rules of Court, though technically legislation, are also
dealt with separately: see \ref{courtrules}.

\subsection{UK Primary Legislation}

\index[general]{legislation!UK!fields|(}
\index[general]{legislation!English|see{UK}}
\index[general]{legislation!Scottish|see{UK}}
\index[general]{legislation!Welsh|see{UK}}
\index[general]{legislation!Northern Irish|see{UK}}
\subsubsection{Bibliographical data}
Primary legislation (that is, UK Acts of Parliament, and the
equivalent measures of the Scottish and Welsh Assemblies and the
Northern Ireland Parliament) are cited using the \texttt{@legislation}
type, setting the \texttt{entrysubtype} to \texttt{primary}. The
relevant fields are as follows.
\begin{description}
\item[entrysubtype] Should always be set to \texttt{primary} for
  primary legislation.
\item[keywords]
\index[general]{keywords field@\texttt{keywords} field}
\index[general]{bibfile@\texttt{.bib} file!keywords field@\texttt{keywords} field}
Should be set to the relevant originating jurisdiction:
  \texttt{gb} or \texttt{en} for legislation made at Westminster,
  \texttt{sc} \texttt{cy} or \texttt{ni} for legislation made in
  Scotland, Wales or Northern Ireland.
\item[title]
\index[general]{title field@\texttt{title} field!legislation@in legislation}
\index[general]{bibfile@\texttt{.bib} file!shorttitle field@\texttt{shorttitle} field} 
  Give the full title of the Act, if it has one, including
  the word `Act'.
\item[shorttitle]
\index[general]{shorttitle field@\texttt{shorttitle} field!legislation@in legislation}
\index[general]{bibfile@\texttt{.bib} file!shorttitle field@\texttt{shorttitle} field}
Give any short title so obvious that you do not
  think it will need explanation. Include a year. (For legislation, I
  strongly recommend that you use the \texttt{shorthand} field
  instead.
\item[shorthand]
\index[general]{shorthand field@\texttt{shorthand} field!legislation@in legislation}
\index[general]{bibfile@\texttt{.bib} file!shorthand field@\texttt{shorthand} field}
\index[general]{legislation!UK!shortened forms}
Give any short title that you want used in second and
  subsequent citations. Note that \oscolashort\ requires that this
  include a year. It is your job to include the
  year; the package will not add it for you.
\item[pagination]
\index[general]{pagination field@\texttt{pagination} field!legislation@in legislation}
\index[general]{bibfile@\texttt{.bib} file!pagination field@\texttt{pagination} field}
Specify how the \emph{main body} of the legislation
  is cited (normally \texttt{section}, for primary legislation).
\item[number]
\index[general]{number field@\texttt{number} field!legislation@in legislation}
\index[general]{bibfile@\texttt{.bib} file!number field@\texttt{number} field}
 Give the number of the legislation. This is only
  strictly required for acts that do not have a short title, and are
  therefore cited by number alone. (In practice that means a few very
  old acts.) For more recent legislation it will normally be
  ignored. It is also required for Scottish, Welsh or Northern Irish
  legislation.
\item[note]
\index[general]{note field@\texttt{note} field!legislation@in legislation}
\index[general]{bibfile@\texttt{.bib} file!note field@\texttt{note} field}
This field can be used if you need to give additional
  explanations (for instance to record that the citation is to
  legislation as amended).
\end{description}
\index[general]{legislation!UK!fields|)}

\subsubsection{Citation}

\index[general]{legislation!citation commands|(}
\index[general]{citation commands!legislation}
Citation commands normally follow the usual pattern, so
\verb|\cite[2]{ucta}| could be used to cite section 2 of the Unfair
Contract Terms Act 1977 (see the examples below). There are, however,
a few complexities, all to do with maintaining a proper index.

First, always make sure that when you cite within the primary `run' of
an Act of Parliament or any other legislation, you cite using the automated addition of section
marks, as above, and not using \verb|\cite[s 2]{ucta}| The first reason
this matters is that \emph{alphabetically} `s 12' comes before `s
2': so a citation like that given above will produce a mis-ordered
index. The second reason is that unless you use the automated system,
the program has no way of extracting subsections to make sure that the
index is properly ordered not only by section but by subsection.

Still, there will be times when you find that you have to cite
something other than a section--for instance a paragraph in a
schedule, and in those cases you will have to format the
\texttt{postnote} manually. The unknown quantity is whether this will
produce a suitable index. If it's just one or two citations, you might
be lucky. If it's more, you will probably need to use \emph{two}
citations: one---easy---to print the citation without indexing it, and
separately a rather elaborate command to make sure the index is correct: see section \ref{trickyindexing} above.
\index[general]{legislation!citation commands|)}

\subsubsection{Examples}
\index[general]{legislation!UK!examples|(}

\begin{bibexample}[ucta]
\begin{verbatim}
@legislation{ucta,
  title        = {Unfair Contract Terms Act},
  date         = {1977},
  entrysubtype = {primary},
  pagination   = {section},
  keywords     = {en},
  number       = {23 Eliz II, cap 23}}
\end{verbatim}
\end{bibexample}

\begin{bibexample}[nia1965]
\begin{verbatim}
@legislation{nia1965,
  title        = {Nuclear Installations Act},
  date         = {1965},
  shorthand    = {NIA 1965},
  keywords     = {en},
  pagination   = {section},
  entrysubtype = {primary},
  keywords     = {en},
}
\end{verbatim}
\end{bibexample}

\begin{description}
\item[\egcite{[2]\{ucta\}}] \cite[2]{ucta}
\item[\egcite{\{nia1965\}}] \cite{nia1965}
\item[\egcite{[4]\{nia1965\}}] \cite[4]{nia1965}
\end{description}

It is also possible to cite Scottish, Welsh and Northern Irish legislation. Set the keywords appropriately, and include the correct reference in the number field.

\begin{bibexample}[learner08]
\begin{verbatim}
@legislation{learner08,
  title          = {Learner Travel (Wales) Measure},
  date           = {2008},
  number         = {nawm 2},
  keywords       = {cy},
  entrysubtype   = {primary},
  pagination     = {section},
}
\end{verbatim}
\end{bibexample}

\begin{description}
\item[\egcite{\{learner08\}}] \cite{learner08}
\end{description}
\index[general]{legislation!UK!examples|)}


\subsubsection{Indexing}
\index[general]{indexing!legislation!primary}
\index[general]{legislation!indexing!UK}
Primary legislation is indexed through one of the five virtual indexes
(one each of England, Wales, Scotland and Northern Ireland, and one
for the UK) reserved for primary legislation. The indexing always
attempts to pick-up pinpoint citations and make individual entries
with sub-levels. The indexes are ordered alphabetically.

\subsection{UK Secondary Legislation}
\index[general]{legislation!UK|(}

\subsubsection{Bibliographic data}

Follow esssentially the same pattern as for UK primary legislation,
giving the SI number in the \texttt{number} field, but setting the
\texttt{entrysubtype} to \texttt{secondary}. Pay attention to
recording the pagination for the instrument in question, since there
is considerable variation.

\begin{bibexample}[disorderly]
\begin{verbatim}
@legislation{disorderly,
  title          = {Penalties for Disorderly Behaviour 
                    (Amendment of Minimum Age) Order},
  date           = {2004},
  entrysubtype   = {secondary},
  pagination     = {regulation},
  keywords       = {en},
  number         = {SI 2004\slash 3166},
  }
\end{verbatim}
\end{bibexample}

\begin{bibexample}[hollowware]
\begin{verbatim}
@legislation{hollowware,
  title          = {Hollow-ware and Galvanising Welfare Order},
  date           = {1921},
  number         = {SR~\&~O 1921\slash 2032},
  pagination     = {rule},
}
\end{verbatim}
\end{bibexample}

\begin{bibexample}[eggs]
\begin{verbatim}
@legislation{eggs,
  title          = {Eggs and Chicks (England) Regulations},
  date           = {2009},
  number         = {SI 2009\slash 2163},
  pagination     = {regulation},
  keywords       = {en},
}
\end{verbatim}
\end{bibexample}

\begin{description}
\item[\egcite{\{disorderly\}}] \cite{disorderly}
\item[\egcite{[2]\{hollowware\}}] \cite[2]{hollowware}
\item[\egcite{[6]\{eggs\}}] \cite[6]{eggs}
\end{description}

\subsubsection{Indexing}
\index[general]{indexing!legislation!secondary}
\index[general]{legislation!indexing!UK}
Secondary legislation is indexed through one of the five virtual indexes
(one each of England, Wales, Scotland and Northern Ireland, and one
for the UK) reserved for secondary legislation. The indexing always
attempts to pick-up pinpoint citations and make individual entries
with sub-levels. The indexes are ordered alphabetically.
\index[general]{legislation!UK|)}


\subsection{Draft Legislation}
\index[general]{legislation!draft!UK|(}
The package provides specific support for citation of draft UK (ie
`Westminster' primary legislation).

\subsubsection{Fields}

The fields are fairly self-explanatory. The date should be a range,
because it refers to a session of Parliament. The institution should
be set to HC or HL depending on the House in which the bill was
introduced. The number should be the printing. The only possibly
counterintuitive thing is that you should set the entrysubtype to
\verb|primary| not \verb|draft|, and include \verb|draft| as a
\emph{keyword}. (This is because it seems possible, in theory, that
one might need to recognise draft secondary legislation too.)

Note that you should enter the title simple as `X Bill'. The package
will take care of adding HL or HC, and of the particular requirements
that apply to specifying the printing, which differ as between the
House of Lords and the House of Commons.

\begin{bibexample}[confund]
\begin{verbatim}
@legislation{confund,
  title        = {Consolidated Fund Bill},
  date         = {2008/2009},
  institution  = {HC},
  number       = {5},
  keywords     = {gb, draft},
  entrysubtype = {primary},
  pagination   = {clause},
}
\end{verbatim}
\end{bibexample}

\begin{bibexample}[confund2]
\begin{verbatim}
@legislation{confund2,
 title         = {Consolidated Fund Bill},
 date          = {2008/2009},
 institution   = {HC},
 number        = {34},
 keywords      = {gb, draft},
 entrysubtype  = {primary},
 pagination    = {clause},
}
\end{verbatim}
\end{bibexample}

\begin{bibexample}[academies]
\begin{verbatim}
@legislation{academies,
  title        = {Academies Bill},
  date         = {2010/2011},
  institution  = {HL},
  number       = {1},
  keywords     = {gb, draft},
  entrysubtype = {primary},
  pagination   = {clause},
}
\end{verbatim}
\end{bibexample}

\begin{description}
\item[\egcite{\{confund\}}] \cite{confund}
\item[\egcite{[3]\{confund2\}}] \cite[3]{confund2}
\item[\egcite{[8(2)]\{academies\}}] \cite[8(2)]{academies}
\end{description}

\index[general]{legislation!explanatory notes}
\textsc{Bl-oscola} also provides for the citation of explanatory notes on draft
legislation. Give this the entry type \verb|@legal|, and the
entrysubtype of \texttt{explanatory note}. It's usually advisable also
to specify an \texttt{indextitle}, so that the document gets indexed
by the Act to which it refers.
\index[general]{legislation!draft!UK}

\begin{bibexample}[charitiesnotes]
\begin{verbatim}
@legal{charitiesnotes,
 title         = {Explanatory Notes on the Charities Act 2006},
 indextitle    = {Charities Act 2006, Explanatory Note},
 pagination    = {paragraph},
 entrysubtype  = {explanatory note},
}
\end{verbatim}
\end{bibexample}

\begin{description}
\item[\egcite{[10]\{charitiesnotes\}}] \cite[10]{charitiesnotes}
\end{description}

For advice on citing debates in Hansard, and in Parliamentary
committees, see section \ref{parliamentary}.

\index[general]{legislation!draft!Scottish}
The package does not provide specific facilities for referring to
Scottish or Welsh draft legislation. But if you do need to do this,
you should be able to use the \verb|misc| entry type, perhaps with
some special indexing. For instance:

\begin{bibexample}[spbill4]
\begin{verbatim}
@misc{spbill4,
  title        = {SP Bill 4 Abolition of Feudal Tenure etc (Scotland)
                  Bill [as introduced] Session 1},
  date         = {1999},
}
\end{verbatim}
\end{bibexample}

\begin{description}
\item[\egcite{\{spbill4\}}] \cite{spbill4}
\end{description}

\subsubsection{Indexing}

\index[general]{indexing!draft legislation}
\index[general]{legislation!draft!indexing}
{\sloppy All English and UK draft legislation, is sent to whatever index has been
associated with \verb|gbdraftleg|. Draft legislation is indexed with
sub-entries by clause and paragraph, as if it were
legislation. Explanatory notes are indexed in the \texttt{gbparltmat}
virtual index.\index[general]{legislation!explanatory notes!indexing}}

\subsection{EU Legislation}

\index[general]{legislation!EU|(}Like \oscolashort, \oscola\ includes the \textsc{eu} treaties as a
type of \textsc{eu} legislation. Nevertheless, there are small
differences, so that it is convenient to deal with treaties and other
legislation separately. Note that decisions in what might be called
\enquote{cases} are dealt with a case citations (see section
\ref{eudecisioncases}).

\subsubsection{Treaties}
\index[general]{EU treaties!generally|see{treaties, European Union}}
\index[general]{treaties!European Union|(}

The \textsc{eu} treaty type is intended to be used for the
foundational treaties of the \textsc{eu}. Treaties to which the
\textsc{eu} is \emph{party} should be cited as international treaties
(following the instructions given in section \ref{piltreaties}).

Provide the following information\label{eutreaties}:
\begin{description}
\item[\texttt{keywords}]
\index[general]{keywords field@\texttt{keywords} field!EU treaties@in EU treaties}
\index[general]{bibfile@\texttt{.bib} file!keywords field@\texttt{keywords} field}
Should be set to \texttt{eu}.
\item[\texttt{entrysubtype}] Should be set to \texttt{eu-treaty}.
\index[general]{entrysubtype field@\texttt{entrysubtype} field!EU treaties@in EU treaties}
\index[general]{bibfile@\texttt{.bib} file!entrysubtype field@\texttt{entrysubtype} field}
\item[\texttt{title}] The full title of the treaty in question.
\index[general]{title field@\texttt{title} field!EU treaties@in EU treaties}
\index[general]{bibfile@\texttt{.bib} file!title field@\texttt{title} field}
\item[\texttt{date}] 
\index[general]{date field@\texttt{date} field!EU treaties@in EU treaties}
\index[general]{bibfile@\texttt{.bib} file!date field@\texttt{date} field}
The date of publication in the source to which
  you are referring (usually and correctly the Official Journal).
\item[\texttt{journaltitle}]
\index[general]{journaltitle field@\texttt{journaltitle} field!EU treaties@in EU treaties}
\index[general]{bibfile@\texttt{.bib} file!journaltitle field@\texttt{journaltitle} field}
 The publication in which the treaty appears
  (usually \texttt{OJ}).
\item[\texttt{series}]
\index[general]{series field@\texttt{series} field!EU treaties@in EU treaties}
\index[general]{series field@\texttt{series} field!Official Journal}
\index[general]{bibfile@\texttt{.bib} file!series field@\texttt{series} field}
In the case of the Official Journal, whether
  the publication is in the L or C series (for treaties it will
  usually be in the C series; legislative acts will normally be in
  the L series).
\item[\texttt{volume} or \texttt{issue}]
\index[general]{volume field@\texttt{volume} field!EU treaties@in EU treaties}
\index[general]{bibfile@\texttt{.bib} file!volume field@\texttt{volume} field} The volume of the series in
  question (in the case of the Official Journal you may also use
  \texttt{issue}).
\item[\texttt{pages}]
\index[general]{pages field@\texttt{pages} field!EU treaties@in EU treaties}
\index[general]{bibfile@\texttt{.bib} file!pages field@\texttt{pages} field}
 At least the first page on which the document starts.
\item[\texttt{shorthand}]
\index[general]{shorthand field@\texttt{shorthand} field!EU treaties@in EU treaties}
\index[general]{bibfile@\texttt{.bib} file!shorthand field@\texttt{shorthand} field}
Any short form of the document you will use
  for subsequent citation which you want explicitly introduced (eg
  \texttt{TEU} or \texttt{TFEU}). This is, of course, optional.
\item[\texttt{shorttitle}]
\index[general]{shorttitle field@\texttt{shorttitle} field!EU treaties@in EU treaties}
\index[general]{bibfile@\texttt{.bib} file!shorttitle field@\texttt{shorttitle} field}
Any short form of the document you will use
  for subsequent citation that you \emph{do not} want to be explicitly
  introduced (more common for legislation than for treaties).
\item[\texttt{pagination}]
\index[general]{pagination field@\texttt{pagination} field!EU treaties@in EU treaties}
\index[general]{bibfile@\texttt{.bib} file!pagination field@\texttt{pagination} field}
 The sort of `pagination' the document uses
  by default (usually \texttt{article} to produce `art' and `arts'.
\end{description}

\index[general]{treaties!European Union!example}
Thus:\footnote{\cite[29]{oscola}, updated to reflect the most recent
  publication of the consolidated version of the TEU.}
\begin{bibexample}[teu]
\begin{verbatim}
@legislation{teu,
  keywords       = {eu},
  entrysubtype   = {eu-treaty},
  title          = {Consolidated Version of the 
                    Treaty on European Union},
  date           = {2010},
  journaltitle   = {OJ},
  series         = {C},
  volume         = {83},
  pages          = {13},
  shorthand      = {TEU},
  pagination     = {article},
}
\end{verbatim}
\end{bibexample}

\begin{description}
\item[\egcite{[3]\{teu\}}] \cite[3]{teu}
\end{description}
\index[general]{treaties!European Union|)}

\index[general]{EU treaties!shortened citations}
\index[general]{options!eu-treaty@\texttt{eu-treaty}}
In subsequent citations \oscola\ will, by default, produce `TEU, art
\emph{x}'. If you prefer the format `Art \emph{x} TEU' (very common in
specialist writing), use the bibliography option
\texttt{eutreaty={}alternative} when loading \biblatex. (See section
\ref{options} above.)

\index[general]{treaties!European Union!indexing}
\index[general]{indexing!EU treaties}
References to \textsc{eu} treaties are indexed via the virtual index
\texttt{eutreaty}, organised alphabetically by title. Attempts are
made to pick postnotes apart so as to order references by article and
sub-article.

\subsubsection{Other EU legislation}

\index[general]{legislation!EU!regulations}\index[general]{legislation!EU!directives}\index[general]{legislation!EU!decisions}
Use the fields given for \textsc{eu} treaties in section
\ref{eutreaties} above, with the following changes:

\begin{description}
  \item[title] 
  \index[general]{title field@\texttt{title} field!EU legislation@in EU legislation}
  \index[general]{bibfile@\texttt{.bib} file!title field@\texttt{title} field}
Ensure that the title includes the number and the
  enacting institution(s) and the date. Nice as it would be to
  `construct' the whole title from the other information, it is not
  possible because the institutions are not consistent in how they
  name instruments.
  \item[entrysubtype] 
  \index[general]{entrysubtype field@\texttt{entrysubtype} field!EU legislation@in EU legislation}
  \index[general]{bibfile@\texttt{.bib} file!entrysubtype field@\texttt{entrysubtype} field}
  May be set to \texttt{directive},
  \texttt{regulation} or \texttt{decision}. It will be used for
  indexing only, and need not be set unless you want the instrument to
  be indexed as something other than its apparent type, as
  given in the \texttt{type} field.
  \item[type]
  \index[general]{type field@\texttt{type} field!EU legislation@in EU legislation}
  \index[general]{bibfile@\texttt{.bib} file!type field@\texttt{type} field}  Set this to the type you want the instrument
  \emph{described as}
  in any short citation, eg \texttt{directive} for a directive. 
  (Note lower case.) There is no need to use abbreviations: these 
  will be handled automatically.
  \item[shorttitle] 
  \index[general]{shorttitle field@\texttt{shorttitle} field!EU legislation@in EU legislation}
  \index[general]{bibfile@\texttt{.bib} file!shorttitle field@\texttt{shorttitle} field}
  This is optional. If it is left blank,
  \oscola\ will construct a short reference using the type and number.
  \item[number] 
  \index[general]{number field!\texttt{number} field!EU legislation@in EU legislation}
  \index[general]{bibfile@\texttt{.bib} file!number field@\texttt{number} field}
  Include the number as it appears in the title of
  the instrument. It is needed to construct a short title, and, even
  if you provide your own, to make sure the instrument is properly
  indexed.
\end{description}

\index[general]{legislation!EU!type field@\texttt{type} field}
The reason there is a distinction between the \texttt{type} (which is used in
printing the citation) and the \texttt{entrysubtype} (used in indexing) is that
there are some instruments which are \emph{called} `Decisions' (or
something else) but function as if they were regulations: logic
dictates that they are indexed as regulations
nonetheless.\footcite[The curious can consult][763 n 502]{lenaerts2}

The first citation will produce the full title. Subsequent citations
will simply use a shorthand (if it has been defined) or short title
(if it has been defined), or will construct a title from the
\texttt{type} and \texttt{number}.

So, for example:
\begin{bibexample}[2002/60]
\begin{verbatim}
@legislation{2002/60,
  title           = {Council Directive 2002\slash 60\slash 
                     EC of 27 June 2002 laying down specific 
                     provisions for the control of African 
                     swine fever and amending Directive  
                     99\slash 119\slash EEC as regards Teschen  
                     disease and African swine fever},
  date            = {2002},
  origdate        = {2002-06-27},
  journaltitle    = {OJ},
  series          = {L},
  issue           = {192},
  pages           = {27},
  type            = {directive},
  number          = {2002/60/EC},
  pagination      = {article},
  keywords        = {eu},
}
\end{verbatim}
\end{bibexample}

{\sloppy
\begin{description}
\item[\egcite{\{2002/60\}}] \cite{2002/60}
\item[\egcite{\{teu\}}] \cite{teu}
\item[\egcite{[2]\{2002/60\}}] \cite[2]{2002/60}
\end{description}}

\index[general]{indexing!EU legislation}
Separate virtual indices are produced for regulations (or
regulation-like instruments), directives (or directive-like
instruments) and decisions (or decision-like instruments). They are
ordered chronologically and by number.

\index[general]{legislation!draft!EU}
Draft \textsc{eu} legislation is cited using the \texttt{@report}
entry type, described in section \ref{drafteu}.

\section{Court Rules}

\index[general]{court rules of@court, rules of|(}
\index[general]{rules of court|(}
\index[general]{procedure rules|(}
The\label{courtrules} \oscola\ standard allows you to cite court rules
(the Civil Procedure Rules, the Rules of the Supreme Court
and the County Court Rules) in a special way. (I'm not sure it's
mandatory to do so. And I don't see why the same couldn't apply the
the Crown Court Rules, but apparently it doesn't.)

In effect, the rules provide for four special cases, and the package
follows suit. It's probably best not to try to understand them. Just
know that if you do what you are told, you will get the right result.

The relevant entries should be defined in your bibliography database
as \texttt{legislation} with the \texttt{entrysubtype} set to
\texttt{procedure-rule}. In the case of the CPR and Practice
Directions made under them, citations are `naked' (without any `r' or
`para') and \texttt{pagination} is left blank. The RSC and CCR are
cited by rule, so pagination is set to \texttt{rule}:

\begin{bibexample}[cpr] 
\begin{verbatim}
@legislation{cpr, 
title        = {Civil Procedure Rules}, 
shorttitle   = {CPR}, 
keywords     = {en}, 
entrysubtype = {procedure-rule},
} 
\end{verbatim} \end{bibexample}
\begin{bibexample}[pd] 
\begin{verbatim} 
@legislation{pd, 
title         = {CPR Practice Direction}, 
shorttitle    = {PD}, 
keywords      = {en}, 
entrysubtype  = {procedure-rule},
}
\end{verbatim}
\end{bibexample}
\begin{bibexample}[rsc] 
\begin{verbatim} 
@legislation{rsc, 
title        = {Rules of the Supreme Court},
shorttitle   = {RSC}, 
keywords     = {en},
pagination   = {rule}, 
entrysubtype = {procedure-rule},
} 
\end{verbatim}
\end{bibexample} 
\begin{bibexample}[ccr] 
\begin{verbatim}
@legislation{ccr, 
title        = {County Court Rules}, 
shorttitle   = {CCR},
keywords     = {en}, 
pagination   = {rule}, 
entrysubtype = {procedure-rule}, }
\end{verbatim} \end{bibexample} Note that strictly speaking the
\texttt{title} field is not essential; it will not be printed in
citations (but will be used simply for indexing).

When you come to cite these sources, you need to use a slightly
different format from the usual. Your postnote should be divided into
two parts, separated by | if necessary, corresponding to the order (or
part of the CPR, or Practice Direction) and the rule (or paragraph)
you wish to cite.

\begin{description}
\item[\egcite{[7]\{cpr\}}] \cite[7]{cpr}
\item[\egcite{[24|14A]\{rsc\}}] \cite[24|14A]{rsc}
\item[\egcite{[17|11]\{ccr\}}] \cite[17|11]{ccr}
\item[\egcite{[6|4.1]\{pd\}}] \cite[6|4.1]{pd}
\end{description}

Using these postnote fields, rather than doing
\verb|\cite[Ord 11, r 1]{rsc}| makes sure that your references will
get indexed properly.

\index[general]{court rules of@court, rules of|)}
\index[general]{rules of court|)}
\index[general]{procedure rules|)}
\index[general]{legislation|)}

\section{Treaties\label{piltreaties}}
\index[general]{treaties|(}

\subsection{Basic use}

Treaties use the following fields:
\begin{description}
\item[entrysubtype]
\index[general]{entrysubtype field@\texttt{entrysubtype} field!treaties@public international law treaties}
\index[general]{bibfile@\texttt{.bib} file!entrysubtype field@\texttt{entrysubtype} field}
This should be set to \texttt{piltreaty}.
\item[title] 
\index[general]{title field@\texttt{title} field!treaties@in treaties}
\index[general]{bibfile@\texttt{.bib} file!title field@\texttt{title} field}
The full title of the treaty.
\item[shorttitle] 
\index[general]{shorttitle field@\texttt{shorttitle} field!treaties@in treaties}
\index[general]{bibfile@\texttt{.bib} file!shorttitle field@\texttt{shorttitle} field}
A shorter title of the treaty for use in subsequent
  citations (optional).
\item[indextitle]
  \index[general]{indextitle field@\texttt{indextitle} field!treaties@in treaties}
  \index[general]{bibfile@\texttt{.bib} file!indextitle field@\texttt{indextitle} field}
  \index[general]{indexing!treaties}
  A different form of title for use in the index: for
  instance you might want to index `International Agreement on the
  Scheldt' as `Scheldt, International Agreement'. (This is, of course, a standard field, but it is likely to be particularly useful in relation to treaties.)
\item[parties]
  \index[general]{parties field@\texttt{parties} field}
  \index[general]{bibfile@\texttt{.bib} file!parties field@\texttt{parties} field} 
  A list of the States party to the treaty, separated by
  `and'. If there are more than three parties, they will not be
  printed.
\item[execution]
  \index[general]{execution field@\texttt{execution} field}
  \index[general]{bibfile@\texttt{.bib} file!execution field@\texttt{execution} field}
  Information about the various relevant dates: this is
  the only tricky thing about treaties, and is explained below.
\item[\textrm{report information}] The usual information about the
  \texttt{journaltitle}, \texttt{volume}, \texttt{series} and
  \texttt{pages}
\item[pagination] The way the document is referred to (usually
  \texttt{article}).
\item[note]
  \index[general]{note field@\texttt{note} field!treaties@in treaties}
  \index[general]{bibfile@\texttt{.bib} file!note field@\texttt{note} field}
  This can be used to give extra information, for instance
  to record that you are citing a treaty in its amended form.
\end{description}

\subsection{Dates}

\index[general]{dates!treaties@in treaties}
  \index[general]{execution field@\texttt{execution} field}
  \index[general]{bibfile@\texttt{.bib} file!execution field@\texttt{execution} field}
Treaties often need a variety of date references. This is done by
\emph{listing} relevant dates in the \texttt{execution} field. The
package recognises four possible types of date: 
\begin{description}
\item[signed]
\index[general]{treaties!date!signature@of signature}
The date of signature.  
\item[opened] 
\index[general]{treaties!date!opened for signature}
The date on which
a treaty was opened for signature.  
\item[adopted] 
\index[general]{treaties!date!adopted}
The date on which a
treaty was adopted.  
\item[inforce]
\index[general]{treaties!date!force@entered into force} 
The date on which a treaty entered
into force.  
\end{description} 
Give as many of these dates as you
wish, with \texttt{and} between them, in the following format:
\begin{center} \texttt{opened=2010-01-10 and inforce=2011-04-23}
\end{center}

The package is not `clever' about reordering dates: they will be
printed in the order you give them.

If a list of dates is included in the \texttt{execution} field, then
\oscola\ will not print anything from the \texttt{date} field, unless
the year is an essential part of the citation of the report. If the
\texttt{execution} field is empty, but there is a date, then the date
will be printed, without any narrative.

\subsection{Examples}
\index[general]{treaties!examples|(}

\begin{bibexample}[scheldt]
\begin{verbatim}
@legal{scheldt,
  title        = {International Agreement on the Scheldt},
  indextitle   = {Scheldt, International Agreement on}
  parties      = {Belgium and France and Netherlands},
  execution    = {signed=2002-12-01 and inforce=2005-12-01},
  pagination   = {article},
  volume       = {2351},
  journaltitle = {UNTS},
  pages        = {13},
  entrysubtype = {piltreaty},
}
\end{verbatim}
\end{bibexample}

\begin{bibexample}[aaland]
\begin{verbatim}
@legal{aaland,
  title        = {Convention Relating to the Non-Fortification
                  and Neutralisation of the Aaland Islands},
  parties      = {{Aaaland Islands} and Germany and Denmark and Estonia
                  and Finland and France},
  lista        = {opened=1921-10-02 and inforce=1922-12-06},
  pagination   = {article},
  volume       = {9},
  journaltitle = {LNTS},
  pages        = {211},
  entrysubtype = {piltreaty},
}
\end{verbatim}
\end{bibexample}

\begin{bibexample}[mongolia]
\begin{verbatim}
@legal{mongolia,
  title       = {Agreement with Mongolia (Economic Co-operation)},
  shortitle   = {Mongolia Cooperation Agreement},
  journaltitle= {O.J.},
  series      = {L},
  issue       = {264},
  pages       = {1},
  date        = {1978},
  entrysubtype= {piltreaty},
}
\end{verbatim}
\end{bibexample}

\begin{description}[leftmargin=3.6cm,labelwidth=3.4cm,labelsep=0.2cm]
\item[\egcite{\{scheldt\}}] {\sloppy \cite{scheldt}}
\item[\egcite{[1]\{aaland\}}] \cite[1]{aaland}
\item[\egcite{\{mongolia\}}] \cite{mongolia}
\end{description}
\index[general]{treaties!examples|)}

The `note' form can be used where it is necessary to give additional information, such as about amendments.

\index[general]{European Convention on Human Rights}
\index[general]{European Convention on Human Rights|see {also, European Court of Human Rights}}
\index[general]{ECHR|see {European Convention on Human Rights, European Court of Human Rights}}
\index[general]{treaties!European Convention on Human Rights}
\begin{bibexample}[echr:treaty]
\begin{verbatim}
@legal{echr:treaty,
  title        = {Convention for the Protection of Human Rights 
                  and Fundamental Freedoms},
  execution    = { opened=1950-11-04 and inforce=1953-09-03 },
  volume       = {213},
  reporter     = {U.N.T.S.},
  pages        = {221},
  note         = {as amended by Protocol No 14bis to the Convention 
                  for the Protection of Human Rights and
				  Fundamental Freedoms 
                  (opened for signature 27 May 2009, entered into force
                   1 September 2009) CETS No 204},
  shorthand    = {ECHR},
  entrysubtype = {piltreaty},
}
\end{verbatim}
\end{bibexample}

\begin{description}[leftmargin=3.6cm,labelwidth=3.4cm,labelsep=0.2cm]
\item[\egcite{\{echr:treaty\}}] \cite{echr:treaty}
\end{description}

It's not, of course, ideal that details of amendments and the like are
not automatically formatted; but although this feature may be
introduced in a future release, for the present the approach taken
here will probably do in most cases.
\index[general]{treaties|)}

\section{Books}
\index[general]{books|(}

\subsection{Basic Use}

Books function exactly as in regular \biblatex. They should be given
the \texttt{@book} type. The most commonly used fields are as follows:
\begin{description}
\item[title] 
  \index[general]{title field@\texttt{title} field!books@in books}
  \index[general]{bibfile@\texttt{.bib} file!title field@\texttt{title} field}
  The title of the book. \textsc{Oscola} requires that this
  be given with significant words capitalized, and it is recommended
  that you enter it in this way in the database. (If other styles you
  use require only the first letter to be capitalized, the title can
  without too much difficulty be converted; but it is not practically
  possible partially to capitalize an initially uncapitalized title.)
\item[subtitle]
  \index[general]{subtitle field@\texttt{subtitle} field!books@in books}
  \index[general]{bibfile@\texttt{.bib} file!subtitle field@\texttt{subtitle} field}
  The subtitle, if any. This will be used only in the
  first or full citation (where it will be printed after the title,
  separated by a colon). (See example \ref{hart08}.)
\item[author]
  \index[general]{author field@\texttt{author} field!books@in books}
  \index[general]{bibfile@\texttt{.bib} file!author field@\texttt{author} field} 
  The author or authors of the work. If there is more than
  one author, their names should be given separated by `and'. It's up
  to you whether you prefer to use the form \texttt{Frederick Bloggs}
  or \texttt{Bloggs, Frederick}---either will work. Since
  \oscolashort\ requires full names, where available, to be given in
  citations, you should give the full name if you have it. No more than the required number of names will be printed;
  any surplus will be abbreviated to `and others', as
  \oscolashort\ requires.\footnote{You can also use `and others'
    yourself, if you like; but it's good practice to include as many
    names as you know, since different styles may have different ideas
    about when to abbreviate, and the database file is supposed to be
    agnostic about such matters.}
\item[editor]
  \index[general]{editor field@\texttt{editor} field!books@in books}
  \index[general]{bibfile@\texttt{.bib} file!editor field@\texttt{editor} field} 
  An editor or editors of the work, if there is one. In some cases,
  if there is only an editor but no author (for instance, the editor
  of a collection of essays) this is used as the alternative to the
  author, in other cases it is additional information. The formatting
  and printing requirements under \oscolashort\ are different, but the
  style takes care of that. Use the same format for names as in the
  \texttt{author} field.
\item[translator]
  \index[general]{translator field@\texttt{title} field}
  \index[general]{bibfile@\texttt{.bib} file!translator field@\texttt{translator} field} 
  Like the \texttt{editor} field, and used in the same
  way.
\item[date]
  \index[general]{date field@\texttt{date} field!books@in books}
  \index[general]{bibfile@\texttt{.bib} file!date field@\texttt{date} field} 
  The date of publication. Where you are using a modern edition of an old work, give the date of the edition to which you are referring: the original date of publication can be given in the \texttt{origdate} field.
\item[origdate]
  \index[general]{origdate field@\texttt{origdate} field}
  \index[general]{bibfile@\texttt{.bib} file!origdate field@\texttt{origdate} field} 
  Used where you are referring to a modern edition of an
  old work, and want to give the original date of publication in
  addition to the date of publication of the particular edition that
  you are citing.
\item[edition]
  \index[general]{edition field@\texttt{edition} field}
  \index[general]{bibfile@\texttt{.bib} file!edition field@\texttt{edition} field} 
  Where possible give the edition number as a number:
  \texttt{\{5\}} rather than \texttt{\{5th\}}.
\item[publisher]
  \index[general]{title field@\texttt{publisher} field}
  \index[general]{bibfile@\texttt{.bib} file!publisher field@\texttt{publisher} field} 
  One or more publishers (if there is more than one,
  separate them with \texttt{and}).
\item[location]
  \index[general]{location field@\texttt{location} field!books@in books}
  \index[general]{bibfile@\texttt{.bib} file!location field@\texttt{location} field}
  \index[general]{publication, place of} 
  The place(s) of publication. In general, this is not
  used if there is a publisher for works after 1800; for works
  before 1800 it is used in preference to the publisher. If you want
  to change the point at which this change occurs, change the value of
  \verb|\bibyearwatershed|.\index[general]{bibyearwatershed@\texttt{\textbackslash bibyearwatershed}} So, for
  instance,
  \begin{verbatim}
\renewcommand{\bibyearwatershed}{1900}
  \end{verbatim}will
  cause the location to be used in preference to the publisher for
  books before 1900.\footcite[35]{oscola}
\item[series]
  \index[general]{series field@\texttt{series} field!books@in books}
  \index[general]{bibfile@\texttt{.bib} file!series field@\texttt{series} field} 
  If the book is in a series, and you think the series
  information is sufficiently important to warrant printing (not, for
  instance, if it is just a series title adopted by the publisher for
  marketing reasons), give it here. 
\item[number]
  \index[general]{number field@\texttt{number} field!books@in books}
  \index[general]{bibfile@\texttt{.bib} file!number field@\texttt{number} field} 
  If the book is part of a series, and has a number (or
  some equivalent identifier), give the number here.
\item[volume]
  \index[general]{volume field@\texttt{volume} field!books@in books}
  \index[general]{bibfile@\texttt{.bib} file!volume field@\texttt{volume} field} 
  Use an \texttt{@book} type with \texttt{volume} where
  each volume is a separate publication (for instance, with a
  different date). For multi-volume reference works use the
  \texttt{@mvreference} type, discussed in section \ref{mvreference} below.\footnote{The requirements of \oscolashort\ regarding volumes are not clear to me. I think that the position is that if the volume is, as it were, an `independent' publication, the volume is printed with the publication information -- but for multivolume reference works, it is treated separately. That, at any rate, is the approach I have taken: if you specify a volume for a book or collection, it will be printed as part of the volume information; if you specify it for a multi-volume reference, it gets printed afterwards.}
\item[pubstate] Use this field if you need to give information about
  the publication state of the book, such as `forthcoming'.
\item[note]
    \index[general]{note field@\texttt{note} field!books@in books}
  \index[general]{bibfile@\texttt{.bib} file!note field@\texttt{note} field} 
  Use this field for any additional notes you need to
  provide about the book.
\item[pagination]
  \index[general]{pagination field@\texttt{pagination} field!books@in books}
  \index[general]{bibfile@\texttt{.bib} file!pagination field@\texttt{pagination} field} 
  If references are to something other than pages
  (usually paragraphs), give the pagination scheme here (eg
  `paragraph': see example \ref{bar00} below.
\end{description}

The following examples\footnote{Drawn from \cite[34--35]{oscola}.}
show use of a variety of the basic fields.

\index[general]{books!examples|(}
\begin{bibexample}[endicott09]
\begin{verbatim}
@book{endicott09,
  author       = {Endicott, Timothy},
  title        = {Administrative Law},
  publisher    = {OUP},
  location     = {Oxford},
  date         = {2009},
}
\end{verbatim}
\end{bibexample}

\begin{bibexample}[jones09]
\begin{verbatim}
@book{jones09,
  author      = {Jones, Gareth},
  title       = {Goff and Jones: 
                 The Law of Restitution},
  edition     = {1st Supp, 7th edn},
  publisher   = {{Sweet~\& Maxwell}},
  date        = {2009},
  location    = {London},
}
\end{verbatim}
\end{bibexample}

\begin{bibexample}[bar00]
\begin{verbatim}
@book{bar00,
  author     = {von Bar, Christian},
  title      = {The Common European Law of Torts},
  volume     = {2},
  publisher  = {CH Beck},
  date       = {2000},
  pagination = {paragraph},
}
\end{verbatim}
\end{bibexample}

\begin{description}
\item[\egcite{\{endicott09\}}] \cite{endicott09}
\item[\egcite{\{jones09\}}] \cite{jones09}
\item[\egcite{[234]\{bar00\}}] \cite[234]{bar00}
\end{description}
\index[general]{books!examples|)}

\subsection{Collections and edited works}

\index[general]{collections}
\index[general]{books!edited works}
\index[general]{books!collections of papers}
With collections or edited works, you can make use either of the
\texttt{@book} and \texttt{@inbook} types, or the \texttt{@collection}
and \texttt{@incollection} types. The package does not distinguish
between them.

The \texttt{@book} or \texttt{@collection} type is the correct choice
to enter details about a whole book, rather than just a particular
part of it such as an essay, paper or chapter. Similarly details of
things like translators and so on can be given, as examples
\ref{birks87} and \ref{kotz98} show.

\index[general]{books!examples|(}
\begin{bibexample}[horder00]
\begin{verbatim}
@book{horder00,
  editor     = {Horder, Jeremy},
  title      = {Oxford Essays in Jurisprudence: 
                Fourth Series},
  publisher  = {OUP},
  date       = {2000},
}
\end{verbatim}
\end{bibexample}

\begin{bibexample}[birks87]
\begin{verbatim}
@book{birks87,
  translator = {Birks, Peter and McLeod, Grant},
  title      = {The Institutes of Justinian},
  indextitle = {Institutes of Justinian, The},
  publisher  = {Duckworth},
  date       = {1987},
  location   = {London},
}
\end{verbatim}
\end{bibexample}

\begin{bibexample}[kotz98]
\begin{verbatim}
@book{kotz98,
  author     = {Zweigert, K. and K{\"o}tz, H.},
  title      = {An Introduction to Comparative 
                Law},
  translator = {Weir, Tony},
  edition    = {3},
  publisher  = {OUP},
  date       = {1998},
}
\end{verbatim}
\end{bibexample}

\begin{bibexample}[hart08]
\begin{verbatim}
@book{hart08,
  author     = {Hart, H. L. A.},
  title      = {Punishment and Responsibility},
  subtitle   = {Essays in the Philosophy of Law},
  editor     = {Gardner, John},
  edition    = {2},
  publisher  = {OUP},
  date       = {2008},
}
\end{verbatim}
\end{bibexample}

\begin{description}
\item[\egcite{\{horder00\}}]\cite{horder00}
\item[\egcite{\{birks87\}}] \cite{birks87}
\item[\egcite{\{kotz98\}}] \cite{kotz98}
\item[\egcite{\{hart08\}}] \cite{hart08}
\end{description}
\index[general]{books!examples|)}

\index[general]{bibfile@\texttt{.bib} file!inbook type@\texttt{inbook} type}
\index[general]{bibfile@\texttt{.bib} file!incollection type@\texttt{incollection} type}
The \texttt{@inbook} and \texttt{@incollection} types are intended to
contain records for component parts of a book---a single article or
chapter, for instance, that is self contained.

If you wish to define a single \texttt{@inbook}, without also defining
a separare entry for the larger book, then essentially the same fields
as are given in relation to books apply, but: 
\begin{description}
\item[author]
  \index[general]{author field@\texttt{author} field!books@in books}
  \index[general]{author field@\texttt{author} field!collections@in collections}
  \index[general]{bibfile@\texttt{.bib} file!author field@\texttt{author} field}
Refers to the author of the individual part, not the
collection as a whole.  
\item[editor] 
  \index[general]{editor field@\texttt{editor} field!books@in books}
  \index[general]{editor field@\texttt{editor} field!collections@in collections}
  \index[general]{bibfile@\texttt{.bib} file!editor field@\texttt{editor} field}
In the case of the
\texttt{@incollection} this refers to the editor of the collection
rather than the individual item. (It is assumed that individual items
do not have their own editors.  
\item[bookauthor] 
  \index[general]{bookauthor field@\texttt{bookauthor} field}
  \index[general]{bibfile@\texttt{.bib} file!bookauthor field@\texttt{bookauthor} field}
Refers to the author of the whole book, rather than the individual work.  
\item[booktitle]
  \index[general]{booktitle field@\texttt{booktitle} field}
  \index[general]{bibfile@\texttt{.bib} file!booktitle field@\texttt{booktitle} field}
Refers to the title of the main book, not the individual work.
\item[mainttitle] 
  \index[general]{maintitle field@\texttt{maintitle} field}
  \index[general]{bibfile@\texttt{.bib} file!maintitle field@\texttt{maintitle} field}
Refers to the title of the main book, not the
individual work.  
\end{description}
See example \ref{pila10} for an example.

\index[general]{crossref field@\texttt{crossref} field!books@in books}
It is often easier, as in examples \ref{castermans09} and \ref{cartwright09}, to define the main \texttt{@book} or
\texttt{@collection}, and then use cross references for individual
parts. In that case, all you are likely to need in the cross
references is the author and title (since \oscolashort\ does not
require page or chapter references).

\index[general]{books!examples|(}
\begin{bibexample}[pila10]
\begin{verbatim}
@inbook{pila10,
  author   = {Pila, Justine},
  editor   = {Dutton, William H. and Jeffreys, Paul W.},
  booktitle= {World Wide Research: Reshaping the 
              Sciences and Humanities in the 
              Century of Information},
  publisher= {MIT Press},
  date     = {2010},
  title    = {The Value of Authorship in the 
              Digital Environment},
  pages    = {210--231},
}
\end{verbatim}
\end{bibexample}

\begin{bibexample}[castermans09]
\begin{verbatim}
@book{castermans09,
  editor   = {Castermans, AG and others},
  title    = {Ex Libris Hans Nieuwenhuis},
  publisher= {Kluwer},
  date     = {2009},
}
\end{verbatim}
\end{bibexample}

\begin{bibexample}[cartwright09]
\begin{verbatim}
@inbook{cartwright09,
  author   = {Cartwright, John},
  title    = {The Fiction of the \enquote{Reasonable 
              Man}},
  crossref = {castermans09},
}
\end{verbatim}
\end{bibexample}


\begin{description}
\item[\egcite{\{pila10\}}] \cite{pila10}
\item[\egcite{\{cartwright09\}}] \cite{cartwright09}
\end{description}
\index[general]{books!examples|)}

\subsection{Older Works}

\index[general]{books!older works|(}
\index[general]{bibyearwatershed@\texttt{\textbackslash bibyearwatershed}}
For works before the date set by \verb|\bibyearwatershed| (by default,
1800) \oscola\ will use the \verb|location| field rather than the
\verb|publisher| field, if present. For works later than that, it will
prefer the \verb|publisher| field, and ignore the \verb|location|
field. You can change this by redefining \verb|\bibyearwatershed| --
or eliminate it altogether by defining \verb|\bibyearwatershed| as 0.

Secondly, for older works it may be useful to provide an
\verb|origdate|, giving the date of original publication as well as
the details of the particular edition you have
used.\footcite[35--36]{oscola}

\begin{bibexample}[hobbes]
\begin{verbatim}
@book{leviathan,
  author     = {Hobbes, Thomas},
  title      = {Leviathan},
  publisher  = {Penguin},
  date       = {1985},
  origdate   = {1651},
  location   = {Harmondsworth},
}
\end{verbatim}
\end{bibexample}

\begin{description}
\item[\egcite{\{leviathan\}}] \cite{leviathan}
\end{description}
\index[general]{books!older works|)}

\subsection{Books of Authority}
\index[general]{books!authority@of authority|(}
\index[general]{institutional works}
\index[general]{Blackstone}

\textsc{Oscola} allows a special citation form for what it calls `books of authority' (certain English and
institutional commentators).\footcite[See][34]{oscola} It turns out
that there are really two different styles---one for English
commentators (where a short form is printed in Roman type), and one
for Scottish institutional writers, where the title is in italics;
this is detected using the keywords \texttt{en} or \texttt{sc}. Use
the \texttt{@commentary} type. In one case (Blackstone) the volume
number comes first. For this, abuse the prenote format.

Where it is available, the \texttt{shorttitle} will be used in the
text: the longer title will, however, be used in indexing.

\begin{bibexample}[stair]
\begin{verbatim}
@commentary{stair,
  author     = {Stair},
  title      = {Institutions},
  keywords   = {sc},
  }
\end{verbatim}
\end{bibexample}

\begin{bibexample}[colitt]
\begin{verbatim}
@commentary{colitt,
  author     = {Coke},
  title      = {Co Litt},
  keywords   = {en},
}
\end{verbatim}
\end{bibexample}

\begin{bibexample}[blackstone]
\begin{verbatim}
@commentary{blackstone,
  author     = {Blackstone},
  title      = {Commentaries},
  shorttitle = {Bl Comm},
  keywords   = {en},
}
\end{verbatim}
\end{bibexample}

\begin{description}
\item[\egcite{[3][254]\{blackstone\}}]\strut\\ \cite[3][254]{blackstone}
\item[\egcite{[135a]\{colitt\}}] \cite[135a]{colitt}
\item[\egcite{[I, 2, 14]\{stair\}}] \cite[I, 2, 14]{stair}
\end{description}
\index[general]{books!authority@of authority|)}

\subsection{Reference works\label{mvreference}}

\index[general]{books!reference works|(}
Reference works should be given the \texttt{@reference} type. Under
\oscolashort\ very little information gets printed about such a work:
really only the title, edition, date and publisher. It is intended
really for use in relation to works that are so well known as works of
reference that the identification of editors, authors and the like is
regarded as superfluous; in cases where that does not seem right, use
\texttt{@book} instead.

If you need to refer to a particular part of a reference work (such as
an article in an encyclopaedia), use that \texttt{@inreference} entry
type. In this case an author or editor may be printed. This can also
be a convenient way of referring to a particular volume of a reference
work without needing to enter a volume number every time, as example
\ref{halsbury5:57} shows.

\begin{bibexample}[halsbury5]
\begin{verbatim}
@reference{halsbury5,
  title       = {Halsbury's Laws},
  pagination  = {paragraph},
  edition     = {5},
  publisher   = {Butterworths},
}
\end{verbatim}
\end{bibexample}

\begin{bibexample}[halsbury5:57]
\begin{verbatim}
@inreference{halsbury5:57,
  volume       = {57},
  date         = {2010},
  crossref     = {halsbury5},
}
\end{verbatim}
\end{bibexample}

\begin{bibexample}[friedrich68]
\begin{verbatim}
@inreference{friedrich68,
  maintitle   = {International Encyclopaedia of the
                 Social Sciences III},
  date        = {1968}
  title       = {Constitutions and Constitutionalism},
  author      = {Friedrich, C. J.},
  pages       = 319, 
}
\end{verbatim}
\end{bibexample}

\begin{description}
\item[\egcite{[123]\{halsbury5:57\}}] \cite[123]{halsbury5:57}
\item[\egcite{\{friedrich68\}}] \cite{friedrich68}
\end{description}

(Note that, in example \ref{friedrich68} the `volume' of the
encyclopaedia was included in the title. This is a fudge, in order to
get the reference to appear as it does in
\oscolashort.\footcite[36]{oscola} It would have been perfectly
possible to give a volume reference here, but that would have produced
a slighly different format: correct, but not exactly as it is in the
Oscola handbook at this point.): \cite{friedrich68a})
\index[general]{books!reference works|)}
\index[general]{encyclopaedia|see {books, reference}}

\subsection{Looseleaf works}
\index[general]{books!looseleaf}
\index[general]{looseleaf publications}
The \oscolashort\ standard requires very minimal information for
looseleaf works. Give such works the entry type \texttt{reference} or
\texttt{mvreference}, and set the option \texttt{looseleaf=true}. You
will need to include the information required by \oscolashort\ (such
as the precise update of the page to which you refer) in your postnote
manually.

\begin{bibexample}[cross]
\begin{verbatim}
@reference{cross,
  title       = {Cross on Local Government Law},
  options     = {looseleaf}}
\end{verbatim}
\end{bibexample}

\begin{description}
\item[\egcite{\{cross\}}] \cite[para 11]{cross}
\item[\egcite{[para 8--106 (R 30 July 2008)]\{cross\}}] \strut\\ \fullcite[para
  8--106 (R 30 July 2008)]{cross}
\end{description}
\index[general]{books|)}

\section{Articles}
\index[general]{articles|(}
\index[general]{journals!articles in|see {articles}}
\index[general]{periodical literature|see {articles}}

Citation of periodical literature using \oscola\ is essentially the
same as in standard \biblatex, with two truly special cases: articles
in newspapers and case notes. All should be given the entrytype
\texttt{@article}.

\begin{description}
\item[entrysubtype] 
\index[general]{entrysubtype field@\texttt{entrysubtype} field!articles@in articles}
\index[general]{bibfile@\texttt{.bib} file!entrysubtype field@\texttt{entrysubtype} field}
Normally unnecessary, but might be set to \verb|newspaper| or \verb|casenote|: see sections \ref{newspaper:article} and \ref{casenotes} below.
\item[author] The author(s) of the article, as a list of names (see section \ref{namelists} for guidance).
\index[general]{author field@\texttt{author} field!articles@in articles}
\index[general]{bibfile@\texttt{.bib} file!author field@\texttt{author} field}
\item[title] The title of the article.
\index[general]{title field@\texttt{title} field!articles@in articles}
\index[general]{bibfile@\texttt{.bib} file!title field@\texttt{title} field}
\item[journaltitle] The name of the journal -- in most cases abbreviated.
\index[general]{journaltitle field@\texttt{journaltitle} field!articles@in articles}
\index[general]{bibfile@\texttt{.bib} file!journaltitle field@\texttt{journaltitle} field}
\item[volume] 
\index[general]{volume field@\texttt{volume} field!articles@in articles}
\index[general]{bibfile@\texttt{.bib} file!volume field@\texttt{volume} field}
The volume of the journal, if it has one. If it is properly referred to by year alone, then the volume should be omitted, and the date will then be printed in square brackets (see example \ref{craig05}). If you require both year and volume, then use the \verb|option| \texttt{year-essential}.\index[general]{options field@\texttt{options} field!year-essential@\texttt{year-essential}}
\item[issue\slash number] 
This is normally unnecessary, but you can give it in addition if it is required as part of the citation.
\item[date] The date of the relevant issue.
\index[general]{date field@\texttt{date} field!articles@in articles}
\index[general]{bibfile@\texttt{.bib} file!date field@\texttt{date} field}
\item[pages] 
\index[general]{pages field@\texttt{pages} field!articles@in articles}
\index[general]{bibfile@\texttt{.bib} file!pages field@\texttt{pages} field}
The first page (or the pages) of the relevant article. You can give a page range if you like, but only the first page will be used.
\item[crossref] Used in conjunction with the \verb|casenote| sub-type to indicate the case to which a note refers.
\index[general]{crossref field@\texttt{crossref} field!articles@in articles}
\index[general]{bibfile@\texttt{.bib} file!crossref field@\texttt{crossref} field}
\item[options] You might need to set the option \texttt{year-essential}, if the year form an essential part of the citation but there is also a volume.
\end{description}

\index[general]{articles!periodical literature!examples|(}
\begin{bibexample}[craig05]
\begin{verbatim}
@article{craig05,
  author       = {Craig, Paul},
  title        = {Theory {\enquote{Pure Theory}} 
                  and Values in Public Law},
  date         = {2005},
  journaltitle = {P.L.},
  pages        = {440},
}
\end{verbatim}
\end{bibexample}
\begin{description}
\item[\egcite{[441]\{craig05\}}] \cite[441]{craig05}
\end{description}

\begin{bibexample}[young09]
\begin{verbatim}
@article{young09,
  author       = {Young, Alison L.},
  title        = {In Defence of Due Deference},
  date         = {2009},
  journaltitle = {MLR},
  volume       = {72},
  pages        = {554},
}
\end{verbatim}
\end{bibexample}

\begin{description}
\item[\egcite{\{young09\}}] \cite{young09}
\end{description}

\begin{bibexample}[griffith01]
\begin{verbatim}
@article{griffith01,
  author       = {Griffith, J. A. G.},
  title        = {The Common Law and the Political Constitution},
  date         = {2001},
  journaltitle = {L.Q.R.},
  volume       = {117},
  pages        = {42},
}
\end{verbatim}
\end{bibexample}

\begin{description}
\item[\egcite{\{griffith01\}}] \cite{griffith01}
\end{description}

\begin{bibexample}[waldron06]
\begin{verbatim}
@article{waldron06,
  author      = {Waldron, Jeremy},
  title       = {The Core of the Case against Judicial Review},
  date        = {2006},
  volume      = {115},
  journaltitle= {Yale L.J.},
  pages       = {1346},
}
\end{verbatim}
\end{bibexample}

\begin{description}
\item[\egcite{[1350--54]\{waldron06\}}] \cite[1350--54]{waldron06}
\end{description}
\index[general]{articles!periodical literature!examples}

\subsection{Newspaper Articles\label{newspaper:article}}

\index[general]{articles!newspapers@in newspapers}
\index[general]{newspapers!articles}
Where an article has appeared in a newspaper, slightly different styling is required. In those circumstances, use the \verb|entrysubtype| of \verb|newspaper|. You may also need to set a \verb|location| field.
\index[general]{articles!entrysubtype field@\texttt{entrysubtype} field!newspaper@\texttt{newspaper}}

\begin{bibexample}[croft]
\begin{verbatim}
@article{croft,
  title        = {Supreme Court Warns on Quality},
  author       = {Jane Croft},
  journaltitle = {Financial Times},
  location     = {London},
  date         = {2010-07-01},
  pages        = {3},
  entrysubtype = {newspaper},
}
\end{verbatim}
\end{bibexample}

\begin{description}
\item[\egcite{\{croft\}}] \cite{croft}
\end{description}

\subsection{Case Notes\label{casenotes}}

\index[general]{articles!case notes}
\index[general]{cases!notes on}
\index[general]{casenotes}
\index[general]{articles!entrysubtype field@\texttt{entrysubtype} field!casenote@\texttt{casenote}}
\index[general]{crossref field@\texttt{crossref field}!casenotes@in casenotes}
Case notes are short articles commenting on a particular case. They
are distinctive in three ways:\footcite[37]{oscola} (1) if cited immediately after the case
to which they refer and in the same footnote, they are printed merely
as `note' (with no title); (2) they can `inherit' their title from the
case to which they refer (in which case it will be printed in italics, with quotation marks); and (3) they are identified as notes
whenever cited.

All this will happen automatically if you use the \texttt{entrysubtype} of \texttt{casenote}, and set the \texttt{crossref} to the case referred to.
For instance, consider the following example:

\begin{bibexample}[jameel04]
\begin{verbatim}
@jurisdiction{jameel04,
  title        = {Jameel v. Wall Street Journal
                  Europe SPRL},
  shorttitle   = {Jameel v Wall Street Journal},
  number       = {[2006] UKHL 44},
  date         = {2007},
  volume       = {1},
  reporter     = {A.C.},
  pages        = {359},
  options      = {year-essential},
}
@article{rowbottom07,
  entrysubtype = {casenote},
  author       = {Rowbottom, Jacob},
  crossref     = {jameel04},
  date         = {2007},
  journaltitle = {C.L.J.},
  pages        = {8--11},
}
\end{verbatim}
\end{bibexample}

If we just cite the note we get: \cite{rowbottom07}. If we cite both
case and note in a footnote, we get a different
format.\footnote{\cite{jameel04}; \cite[see][]{rowbottom07}.}
\index[general]{articles|)}

\section{Reports}
\index[general]{reports|(}

Reports are relatively straightforward:
\begin{itemize}
\item Some documents entitled `report' are actually books. If they
  have an \textsc{isbn} number they should be treated as books and
  entered into the database accordingly: see \oscolashort\ at
  39. In that case, even if the author is an institution, it should be
  entered in the \texttt{author} field. It is usually desirable, in
  such a case, to enter the \texttt{author} with an extra set of
  braces, to prevent it being turned into `Oxford, U of' in the bibliography!
\begin{bibexample}[franks66]
\begin{verbatim}
@book{franks66,
  author    =  {{University of Oxford}},
  title     =  {Report of Commission of Inquiry},
  date      =  {1966},
  publisher =  {OUP},
  shorthand =  {Franks Report},
}
\end{verbatim}
\end{bibexample}
\item Reports of parliamentary committees (see \cite[40]{oscola})
  require special treatment, which is provided by giving them an
  \texttt{entrysubtype} of \texttt{parliamentary}. Further details are
  given below.
\item Commission Documents (see \cite[41]{oscola}) also require
special treatment, which is provided by giving them an
\texttt{entrysubtype} of \texttt{comdoc}. See below.
\item Other reports---such as Command Papers and Law Commission
  Reports are simply given an \texttt{entrytype} of \texttt{\@report}. 
\end{itemize}

\subsection{The Basic Report}

\index[general]{author field@\texttt{author} field!reports@in reports}
\index[general]{reports!author}
\index[general]{institution field@\texttt{institution} field!reports@in reports}
\index[general]{bibfile@\texttt{.bib} file!author field@\texttt{author} field}
\index[general]{bibfile@\texttt{.bib} file!institution field@\texttt{institution} field}
Use either the \texttt{author} or the \texttt{institution} field for
the institution producing the report. There is a difference. In a full
citation, either will get printed as if the author, so
\begin{verbatim}
@report{bribery313,
  institution   =  {Law Commission},
  title         =  {Reforming Bribery}
  ...
}
\end{verbatim}
will produce `Law Commission, \emph{Reforming Bribery} \ldots, as will
\begin{verbatim}
@report{bribery313,
  author        =  {{Law Commission}},
  title         =  {Reforming Bribery}
  ...
}
\end{verbatim}
But on subsequent citations, whereas a report with an author will get
cited as `Author (n \emph{x})', one which has an institution will be
cited as `Title (n \emph{x})'. It's up to you which seems preferable
in any particular case.

If your report has a number (for instance it is a command paper) you
should include this as the \texttt{number}.\index[general]{number field@\texttt{number} field!reports@in reports}\index[general]{bibfile@\texttt{.bib} file!number field@\texttt{number} field} It's up to you whether you
prefer to use the \texttt{series} field as well (for instance for
command papers), or to include the series information in the number
field.\index[general]{series field@\texttt{series} field!reports@in reports}\index[general]{bibfile@\texttt{.bib} file!series field@\texttt{series} field} Either will work perfectly well so far as formatting is
concerned; a purist would probably prefer to separate number and
series. So
\index[general]{reports!command papers}
\index[general]{command papers|see {reports}}
\begin{bibexample}[capital]
\begin{verbatim}
@report{capital,
  institution   = {Home Office},
  title         = {Report of the Royal Commission on
                   Capital Punishment},
  series        = {Cmd},
  number        = {8932}
  ...
}
\end{verbatim}
\end{bibexample}
\noindent will produce the same result as
\begin{bibexample}[capital2]
\begin{verbatim}
@report{capital,
  ...
  series        = {},
  number        = {Cmd 8932}
  ...
}
\end{verbatim}
\end{bibexample}

\index[general]{type field@\texttt{type} field!reports@in reports}
\index[general]{bibfile@\texttt{.bib} file!type field@\texttt{type} field}
\index[general]{reports!white papers}
\index[general]{reports!green papers}
\index[general]{white paper|see {reports}}
\index[general]{green paper|see {reports}}
If you need to specify the type of report (eg `Green Paper', or `White
Paper'), put it in the \texttt{type} field (see example \ref{eliminating:poverty}).

\subsubsection{Examples}

\index[general]{reports!examples|(}
\begin{bibexample}[rc:capital3]
\begin{verbatim}
@report{rc:capital,
  title        = {Report of the Royal Commission on 
                  Capital Punishment},
  institution  = {Home Office},
  series       = {Cmd},
  number       = {8932},
  date         = {1953},
  pagination   = {paragraph},
}
\end{verbatim}
\end{bibexample}

\begin{description}
\item[\egcite{[53]\{rc:capital\}}] \cite[53]{rc:capital}
\end{description}

\begin{bibexample}[autumnperf]
\begin{verbatim}
@report{autumnperf,
  title        = {2008 Autumn Performance Report},
  institution  = {Department for Children, Schools 
                   and Families},
  series       = {Cm},
  number       = {7507},
  date         = {2008},
}
\end{verbatim}
\end{bibexample}

\begin{description}
\item[\egcite{[20]\{autumnperf\}}] \cite[20]{autumnperf}
\end{description}

\begin{bibexample}[eliminating:poverty]
\begin{verbatim}
@report{eliminating:poverty,
  title        = {Eliminating World Poverty},
  subtitle     = {Building our Common Future},
  author       = {{Department for International 
                  Development}},
  number       = {Cm 7656},
  type         = {White Paper},
  date         = {2009},
}
\end{verbatim}
\end{bibexample}

{\sloppy\begin{description}
\item[\egcite{[ch\~5]\{eliminating:poverty\}}] \strut\\\cite[ch~5]{eliminating:poverty}
\end{description}}

\begin{bibexample}[lawcom313]
\begin{verbatim}
@report{lawcom313,
  title        = {Reforming Bribery},
  institution  = {Law Commission},
  number       = {Law Com No 313},
  date         = {2008},
  pagination   = {paragraph},
}
\end{verbatim}
\end{bibexample}

\begin{description}
\item[\egcite{[3.12--3.17]\{lawcom313\}}] \strut\\\cite[3.12--3.17]{lawcom313}
\end{description}

\begin{bibexample}[scotlawcom196]
\begin{verbatim}
@report{scotlawcom196,
  title        = {Damages for Psychiatric Injury},
  institution  = {Scottish Law Commission},
  number       = {Scot Law Com No 196},
  date         = {2004-11-11},
  pagination   = {paragraph},
}
\end{verbatim}
\end{bibexample}

\begin{description}
\item[\egcite{\{scotlawcom196\}}] \cite{scotlawcom196}
\end{description}

\begin{bibexample}[lawcomcp121]
\begin{verbatim}
@report{lawcomcp121,
  title        = {Privity of Contract: Contracts for the Benefit
                  of Third Parties},
  author       = {{Law Commission}},
  number       = {Law Com CP No 121},
  date         = {1991},
  pagination   = {paragraph},
}
\end{verbatim}
\end{bibexample}

\begin{description}
\item[\egcite{\{lawcomcp121\}}] \cite{lawcomcp121}
\end{description}
\index[general]{reports!examples|)}

\subsection{Parliamentary Reports\label{parliamentary}}

\index[general]{reports!parliamentary proceedings@of parliamentary proceedings|(}
You need to distinguish between reports of proceedings in Parliament
(Hansard)\footcite[See][39--40]{oscola} and reports of select and
joint committees.\footcite[See][40]{oscola} Hansard reports are dealt
with using the \texttt{legal} entrytype, and reports of select and
joint committees are dealt with using the \texttt{report} entrytype.

\subsubsection{Hansard}
\index[general]{Hansard}
\index[general]{reports!Hansard}

For proceedings on the floor of either house, the recommended method
of use is as follows.

First,\label{hansard:examples} set up `master' entries for
\emph{Hansard} in the House of Commons and the House of Lords.
\begin{bibexample}[hansardhc]
\begin{verbatim}
@legal{hansardhc,
  title        = {Official Report, House of Commons},
  shorttitle   = {HC Deb},
  entrysubtype = {parliamentary},
  pagination   = {column}
}
\end{verbatim}
\end{bibexample}

Now, for any individual debate that you want to reference, set up an
individual entry along the following lines (for the 2nd reading of the
Bill that became Freedom of Information Act 2000):
\begin{bibexample}[foia:2r]
\begin{verbatim}
@legal{foia:2r,
  volume       = {340},
  pages        = {714--98},
  date         = {1999-12-07},
  crossref     = {hansardhc},
  options      = {skipbib=true},
  
}
\end{verbatim}
\end{bibexample}

\begin{description}
\item[\egcite{[715]\{foia:2r\}}]\cite[715]{foia:2r}
\end{description}

Now you will be able to reference a particular debate, using
\verb|\cite{foia:2r}|. And if you choose to include Hansard in your
bibliography (you probably won't want to), you will not get each
debate printed separately, but just a single entry for
Hansard. Similarly, the index to hansard references will be correct,
with an entry for Hansard, and sub-entries by volume.

For reports of standing committees and joint committees, you will need
to set up a separate database entry, like this:

\begin{bibexample}[healthbill]
\begin{verbatim}
@legal{healthbill,
  title        = {Health Bill Deb},
  date         = {2007-01-30},
  pagination   = {column},
  entrysubtype = {parliamentary},
}
\end{verbatim}
\end{bibexample}

\begin{description}
\item[\egcite{\{healthbill\}}] \cite{healthbill}
\end{description}

When parliamentary materials are cited, \emph{ibid} will be used, but
back references will not: the reference will always be repeated in
full to avoid ambiguity.

\index[general]{indexing!parliamentary materials}
Parliamentary materials are indexed via the \texttt{gbparltmat}
virtual index. If you set up your hansard references as suggested
above, then multiple references will all be placed under the relevant
official report, with sub-items for each volume cited.
\index[general]{reports!parliamentary proceedings@of parliamentary proceedings|)}

\subsubsection{Select and Joint Committee Reports}

\index[general]{reports!select committee}
\index[general]{reports!joint committee}
Select and Joint Committee reports should be entered using the
\texttt{report} entrytype. Use a single number (beginning \texttt{HC}
or \texttt{HL}) for committees of a single house, and both numbers,
separated by commas, for joint committees. The formatting requirements
in either case are rather different, but the package sorts this out.

Although the volume numbers will be printed in roman
numerals, they should be entered in arabic.

The date should be the \emph{session} of the house, given as a range
of dates divided by a slash, as shown in the examples below. The
package will compress the range to comply with the OSCOLA standard.

\subsubsection{Examples}

(For examples of Hansard, see \ref{hansard:examples} above.)

\begin{bibexample}[genomic]
\begin{verbatim}
@report{genomic,
  title        = {Genomic Medicine},
  author       = {Science and Technology Committee},
  entrysubtype = {parliamentary},
  number       = {HL 107},
  volume       = {1},
  date         = {2008/2009},
}
\end{verbatim}
\end{bibexample}

\begin{bibexample}[equality]
\begin{verbatim}
@report{equality,
  title        = {Legislative Scrutiny: Equality Bill
                  (second report); Digital Economy Bill},
  institution  = {Joint Committee on Human Rights},
  entrysubtype = {parliamentary},
  date         = {2009/2010},
  number       = {HL 73, HC 425},
}
\end{verbatim}
\end{bibexample}

\begin{bibexample}[patient]
\begin{verbatim}
@report{patient,
  title        = {Patient Safety},
  institution  = {Health Committee},
  entrysubtype = {parliamentary},
  date         = {2008/2009},
  volume       = {1},
  number       = {HC 151},
  pagination   = {paragraph},
}
\end{verbatim}
\end{bibexample}

\begin{description}
{\sloppy\item[\egcite{\{genomic\}}] \cite{genomic}}
\item[\egcite{[173--75]\{patient\}}]\strut\\ \cite[173--75]{patient}
\item[\egcite{[14--16]\{equality\}}]\strut\\ \cite[14--16]{equality}. 
\end{description}

(Note that the formatting is different
for joint committees, but \oscola\ works that out from the number.)

Such reports are also placed in the \texttt{gbpartlmat} virtual index.

\subsection{EU Commission Documents\label{drafteu}}

\index[general]{comdoc@\textsc{comdoc}|see reports}
\index[general]{reports!comdocs@\textsc{comdoc}s|(}
\index[general]{legislation!draft!EU}
Enter Commission Documents exactly as reports, but with the
\texttt{entrysubtype} field set as \texttt{comdoc}.\index[general]{entrysubtypefield@\texttt{entrysubtype} field!comdocs@for\texttt{comdoc}s}\index[general]{bibfile@\texttt{.bib} file!entrysubtype field@\texttt{entrysubtype} field} Subsequent
references will use simply the \textsc{com} number, as \oscolashort\ requires. If
you need to add a document description to be printed in brackets after
the title, include it in the \texttt{type} field.

The following are the examples from page 41 of \oscola.

\begin{bibexample}[com895]
\begin{verbatim}
@report{com895,
  title        = {Proposal for a Council Decision on the conclusion,
                  on behalf of the European Community, of a Protocol
                  on the Implementation of the Alpine Convention in
                  the Field of Transport (Transport Protocol)},
  institution  = {Commission},
  number       = {COM (2008) 895 final},
  entrysubtype = {comdoc}
}
\end{verbatim}
\end{bibexample}

\begin{bibexample}[com13]
\begin{verbatim}
@report{com13,
  title        = {Action Plan on consumer access to justice 
                  and the settlement of disputes in the internal 
                  market},
  type         = {Communication},
  number       = {COM (96) 13 final},
  institution  = {Commission},
  entrysubtype = {comdoc}
}
\end{verbatim}
\end{bibexample}


\begin{bibexample}[com348]
\begin{verbatim}
@report{com348,
  title        = {Proposal for a Council Regulation on jurisdiction
                  and the recognition and enforcement of judgments 
                  in civil and commercial matters},
  number       = {COM (99) 348 final},
  institution  = {Commission},
  entrysubtype = {comdoc},
}
\end{verbatim}
\end{bibexample}

\begin{description}
\item[\egcite{[ch~1, art~3]\{com895\}}]\strut\\ \cite[ch~1, art~3]{com895}
\item[\egcite{\{com13\}}] \cite{com13}
\item[\egcite{\{com348\}}] \cite{com348}
\end{description}

\index[general]{indexing!EU documents}
EU official documents with the \texttt{comdoc} subtype are currently
indexed by title via the virtual index \texttt{euoffdoc}. This is
likely to be replaced with a system under which they are indexed by
year and number.
\index[general]{reports!comdocs@\textsc{comdoc}s|)}
\index[general]{reports|)}

\section{UN Documents}

\index[general]{United Nations documents|(}
\index[general]{misc entrytype@\texttt{misc} entrytype!UN documents}
The guidance given on citation of \textsc{un} documents in the third edition of \oscolashort\ is
somewhat inconsistent.\footcite[See][32-36]{oscola3} The scheme adopted here has been suggested by Daniel H\"ogger.

United Nations documents should be given the \texttt{entrytype} of \texttt{misc}, and with the \texttt{entrysubtype} set to \texttt{undoc}.

The following fields are used:

\begin{description}
\index[general]{author field@\texttt{author} field!UN documents@in UN documents}
\index[general]{United Nations documents!author}
\index[general]{institution field@\texttt{institution} field!UN documents@in UN documents}
\index[general]{United Nations documents!institution}
\item[author/institution] The body producing the document (eg UNGA for the General Assembly).
\item[title] The `long' title of an instrument, if it has one.
\index[general]{title field@\texttt{title} field!UN documents@in UN documents}
\index[general]{United Nations documents!title}
\item[instrument\_no] The number \emph{of the instrument} (eg Res 251/210). This is different from the
number of the \emph{document}. (Internally this equates to the \texttt{titleaddon} field, which may
by used as an alternative.)
\index[general]{instrument no field@\texttt{instrument\_no} field!UN documents@in UN documents}
\index[general]{United Nations documents!instrument no}
\item[doc\_no] The UN document number. Include the `UN Doc' (since not all documents seem to have it, and \oscola\ therefore
cannot guess when it should be added). Note that for some pre-1976 documents, the date is part of the official document number
and should be added here too. (Internally this equates to the \texttt{number} field, which may be used as an alternative.)
\index[general]{doc no field@\texttt{doc\_no} field!UN documents@in UN documents}
\index[general]{United Nations documents!doc no}
\item[date] The date of the instrument.
\end{description}

You can also use other common fields, such as \texttt{shorttitle}, \texttt{indextitle} and so on. (If you do not
specify a short title, \oscola\ will use the instrument number as the short title.) If you wish to cite a \textsc{un} document 
from a journal or collection (such as ILM), you may include \texttt{journaltitle}, \texttt{volume} and \texttt{pages} fields here.

It is not clear, from the examples given in \oscolashort\ (3rd edition) whether the 
titles of \textsc{un} instruments should be given in quotation marks or not.
By default, \oscola\ does not use quotation marks. But you may, if you wish, alter this. To do so,
include the following in your preamble:
\begin{verbatim}
\DeclareFieldFormat{untitle}{\mkbibquote{#1}}
\end{verbatim}

\index[general]{United Nations documents!examples}
\begin{bibexample}[Indep]
\begin{verbatim}
@misc{Indep,
  author        = {UNGA},
  title         = {Declaration on the Granting of Independence to Colonial
                   Countries and Peoples},
  instrument_no = {Res 1514 (XV)},
  date          = {1960-12-14},
  doc_no        = {UN~Doc A/4684 (1961)},
  shorttitle    = {Res 1514 (XV)},
  entrysubtype  = {undoc},
}
\end{verbatim}
\end{bibexample}

\begin{bibexample}[res51_210]
\begin{verbatim}
@misc{res51_210,
  institution   = {UNGA},
  instrument_no = {Res 51/210},
  date          = {1996-12-17},
  doc_no        = {UN~Doc A/RES/51/210},
  entrysubtype  = {undoc},
}
\end{verbatim}
\end{bibexample}

\begin{bibexample}[badinter4]
\begin{verbatim}
@misc{badinter4,
  title        = {Opinion No 4 of the Badinter Commission},
  indexttitle  = {Badinter Commission, Opinion No 4},
  year         = {1992},
  volume       = {31},
  journaltitle = {I.L.M.},
  pages        = {1501},
  entrysubtype = {undoc},
}
\end{verbatim}
\end{bibexample}

\begin{description}
\item[\egcite{\{Indep\}}] \cite{Indep}
\item[\egcite{\{res51\_210\}}] \cite{res51_210}
\item[\egcite{\{badinter4\}}] \cite{badinter4}
\end{description}

No special index is provided for \textsc{un} documents, but you can use the \texttt{tabulate} field to direct them to an appropriate index if you choose to do so.
\index[general]{United Nations documents|)}


\section{Miscellaneous Other Resources}

There is also support for citation of theses, letters, conference
papers, emails and interviews in accordance with
\oscolashort.\footcite[42--43]{oscola} These hardly warrant detailed
consideration, and so a few examples should suffice.

\index[general]{thesis}
\begin{bibexample}[herberg89]
\begin{verbatim}
@thesis{herberg89,
 title         = {Injunctive Relief for Wrongful 
                  Termination of Employment},
 author        = {Herberg, Javan},
 date          = {1989},
 institution   = {{University of Oxford}},
 type          = {DPhil thesis},
}
\end{verbatim}
\end{bibexample}

\begin{description}
\item[\egcite{\{herberg89\}}] \cite{herberg89}
\end{description}

\index[general]{letter}
\begin{bibexample}[brown:letter]
\begin{verbatim}
@misc{brown:letter,
  title        = {Letter from Gordon Brown 
                  to Lady Ashton},
  date         = {2009-11-20},
}
\end{verbatim}
\end{bibexample}

\begin{bibexample}[amazon:email]
\begin{verbatim}
@misc{amazon:email,
  title        = {Email from Amazon.co.uk to author},
  date         = {2008-12-16},
}
\end{verbatim}
\end{bibexample}

\begin{description}
\item[\egcite{\{brown:letter\}}] \cite{brown:letter}
\item[\egcite{\{amazon:email\}}] \cite{amazon:email}
\end{description}

\index[general]{interview}
\begin{bibexample}[honore:interview]
\begin{verbatim}
@misc{honore:interview,
  author       = {Endicott, Timothy and Gardner, John},
  title        = {Interview with Tony Honor{\'e}, 
                  Emeritus Regius Professor
                  of Civil Law, University of Oxford},
  location     = {Oxford},
  date         = {2007-07-17},
  shorttitle   = {Interview with Tony Honor\'e},
}
\end{verbatim}
\end{bibexample}

\begin{description}
\item[\egcite{\{honore:interview\}}]\strut\\ \cite{honore:interview}
\end{description}

\index[general]{conference}
\index[general]{proceedings}
\index[general]{papers|see {article, conference}}
\begin{bibexample}[reliance]
\begin{verbatim}
@inproceedings{reliance,
  author       = {McFarlane, Ben and Nolan, Donal},
  title        = {Remedying Reliance: The Future Development of
                  Promissory and Proprietary Estoppel in English Law},
  eventtitle   = {Obligations III conference},
  location     = {Brisbane},
  eventdate    = {2006-07-01},
}
\end{verbatim}
\end{bibexample}

\begin{description}
\item[\egcite{\{reliance\}}] \cite{reliance}
\end{description}


\section{Online Materials\label{urls}}
\index[general]{online sources|(}
There are two aspects to the citation of online materials. The
\oscola\ package offers you a specific way of citing material such as
websites. It also offers you the opportunity of providing additional
online information in relation to some other material.

\subsection{Online sources}

For exclusively online material (such as
websites)\footcite[42]{oscola} use the \texttt{@online} entry type.

\begin{description}
\item[title] The title of the piece.
\index[general]{title field@\texttt{title} field!online@in online sources}
\index[general]{bibfile@\texttt{.bib} file!title field@\texttt{title} field}
\item[journaltitle] The equivalent to the title of a journal, for instance the title of the website or blog.
\index[general]{journaltitle field@\texttt{journaltitle} field!online@in online sources}
\index[general]{bibfile@\texttt{.bib} file!journaltitle field@\texttt{journaltitle} field}
\item[author] The author or authors of the piece.
\index[general]{author field@\texttt{author} field!online@in online sources}
\index[general]{bibfile@\texttt{.bib} file!author field@\texttt{author} field}
\item[date] The date of the piece (assuming it is available).
\index[general]{date field@\texttt{date} field!online@in online sources}
\index[general]{bibfile@\texttt{.bib} file!date field@\texttt{date} field}
\item[url] The \textsc{url} address.
\index[general]{url field@\texttt{url} field!online@in online sources}
\index[general]{bibfile@\texttt{.bib} file!url field@\texttt{url} field}
\item[urldate] The date on which you last accessed that \textsc{url}.
\index[general]{urldate field@\texttt{urldate} field!online@in online sources}
\index[general]{bibfile@\texttt{.bib} file!urldate field@\texttt{urldate} field}
\end{description} 

\begin{bibexample}[cole09]
\begin{verbatim}
@online{cole09,
  title        = {Virtual Friend Fires Employee},
  journaltitle = {Naked Law},
  date         = {2009-05-01},
  url          = {http://www.nakedlaw.com/2009/05/index.html},
  urldate      = {2009-11-09},
  author       = {Cole, Sarah},
}
\end{verbatim}
\end{bibexample}

\begin{description}
\item[\egcite{\{cole09\}}] \cite{cole09}
\end{description}

\index[general]{journals!electronic}
For electronic journals, use instead the \texttt{@article} entry type,
giving in addition to the regular information the \texttt{url} and
\texttt{urldate}.  Since \oscolashort\ asks you to cite, as far as
possible, in the style adopted by the journal, you may sometimes need
to be `creative' in the way you use fields to get the citation
correct. For instance, while example \ref{greenleaf10} works out
perfectly using conventional fields, the particular preferences of the
\emph{Duke Law and Technology Review} (for a date without any
brackets, and an issue number with leading zeros) could only be met
by including this information in the \texttt{journaltitle}.

\begin{bibexample}[greenleaf10]
\begin{verbatim}
@article{greenleaf10,
  author       = {Greenleaf, Graham},
  title        = {The Global Development of Free Access 
                  to Legal Information},
  date         = {2010},
  url          = {http://ejlt.org/article/view/17},
  urldate      = {2010-07-10},
  journaltitle = {EJLT},
  volume       = {1},
  issue        = {1},
}
\end{verbatim}
\end{bibexample}

\begin{bibexample}[boyle04]
\begin{verbatim}
@article{boyle04,
  author       = {Boyle, James},
  title        = {A Manifesto on WIPO and 
                  the Future of Intellectual 
                  Property},
  journaltitle = {2004 Duke L \& Tech Review 0009},
  url={http://dltr.law.duke.edu/2004/09/08/a-manifesto-on-
wipo-and-the-future-of-intellectual-property},
  urldate      = {2012-07-18},
  options      = {url=true},
}
\end{verbatim}
\end{bibexample}

\begin{description}
\item[\egcite{\{greenleaf10\}}] \cite{greenleaf10}
\item[\egcite{\{boyle04\}}] \cite{boyle04}
\end{description}
\index[general]{online sources|)}

\subsection{URLs in other material}

\index[general]{urls@\textsc{url}s}
\index[general]{bibfile@\texttt{.bib} file!url field@\texttt{url} field}
In addition to the citation of material that is \emph{primarily}
internet-based, you can also add \textsc{url}s to books or articles,
in which case the \textsc{url} and date will be printed in addition to
the other information. You can also do so with international cases
(since \oscolashort\ expressly provides for this). There is not,
however, any facility for adding \textsc{url}s to other cases, because
even when they are unreported the case name and reference (including
the neutral citation) should enable the reader to look them up on any
of a number of electronic resources.

So, for instance, example \ref{melbourne} shows the citation of a book with a \textsc{url}, and example \ref{loader} shows an article.

\begin{bibexample}[melbourne]
\begin{verbatim}
@book{melbourne,
  title      = {Australian Guide to Legal Citation},
  author     = {{Melbourne University Law Review 
                 Association Inc}},
  publisher  = {Melbourne University Law Review 
                Association and 
                Melbourne Journal of International Law},
  edition    = {3},
  date       = {2010},
  url        = {http://mulr.law.unimelb.edu.au/go/aglc},
  urldate    = {2012-07-18},
}
\end{verbatim}
\end{bibexample}

\begin{description}
\item[\egcite{\{melbourne\}}] \cite{melbourne}
\end{description}

\begin{bibexample}[loader]
\begin{verbatim}
@article{loader,
  title        = {The Great Victim of this Get Tough
                  Hyperactivity is Labour},
  author       = {Ian Loader},
  journaltitle = {The Guardian},
  location     = {London},
  date         = {2008-06-19},
  url={http://www.guardian.co.uk/commentisfree/2008/jun/19
/justice.ukcrime},
  urldate      = {2009-11-19},
  entrysubtype = {newspaper},
}
\end{verbatim}
\end{bibexample}

\begin{description}
\item[\egcite{\{loader\}}] \cite{loader}
\end{description}

\subsubsection{Controlling whether \textsc{url}s are printed}
\index[general]{urls@\textsc{url}s!controlling printing}
\index[general]{options!url@\texttt{url}}
\index[general]{options field@\texttt{options} field!url option@\texttt{url} option}
If you load \biblatex\ with the option \texttt{url=false}, then the
printing of \textsc{url}s will generally be supressed. However, they
will still be printed for sources that have been given the
\texttt{online} type, and you can also override this on an
item-by-item basis, by setting the \texttt{url} option for that
particular source. This is obviously recommended for any source which
does not have the \texttt{online} type, but is only available
electronically. You can see it in example \ref{boyle04}.

\subsection{Formatting of URLs}

\index[general]{urls@\textsc{url}s!formatting}
\index[general]{angled brackets}
\index[general]{legalstarturl@\texttt{\textbackslash legalstarturl}}
\index[general]{legalendurl@\texttt{\textbackslash legalendurl}}
The \oscolashort\ standard requires URLs to be placed in what are
described in the text as angled brackets\footcite[33]{oscola}, but
show not as angled brackets ($\langle\rangle$) but as greater-than and
less-than signs (\textless\textgreater). Since angled brackets are in fact
typographically better, this is what \oscola\ uses: but if you prefer
to do so, you can use other delimeters by redefining\linebreak
\verb|\legalstarturl| and \verb|\legalendurl|. So, for instance:
\begin{bibexample}[delims]
\begin{verbatim}
\renewcommand{\legalstarturl}{\textless}
\renewcommand{\legalendurl}{\textgreater}
\end{verbatim}
\end{bibexample}
will produce
{\begin{quote}
\renewcommand{\legalstarturl}{\textless}
\renewcommand{\legalendurl}{\textgreater}
\fullcite{loader}
\end{quote}}



\clearpage

\addcontentsline{toc}{section}{Appendices}

\appendix

\section{Licence}

\subsection{Main Package}

The \oscola\ package for \biblatex\ consists of the following files:
\begin{itemize}
\item \texttt{oscola.bbx}
\item \texttt{oscola.cbx}
\item \texttt{english-oscola.lbx}
\item \texttt{oscola.ist}
\end{itemize}

This work may be distributed and\slash or modified under the conditions of the \LaTeX\ Project Public License, either version 1.3 of this license or (at your option) any later version. The latest version of the license is in
\begin{center}
\url{http://www.latex-project.org/lppl.txt}
\end{center}
and version 1.3 or later is part of all distributions of \LaTeX\ version 2005\slash 12\slash 01 or later.

This work has the LPPL maintenance status of `maintained'. The current maintainer of this work is Paul Stanley (\texttt{pstanley@essexcourt.net}).

\subsection{Documentation}

The documentation for this package, consists of the following files:
\begin{itemize}
\item \texttt{oscola.pdf}
\item \texttt{oscola-examples.bib}
\end{itemize}

These files are distributed under the Creative Commons Attribution 3.0-Unported License (CC BY 3.0). A copy of that license is available at
\begin{center}
\url{http://creativecommons.org/licenses/by/3.0/deed.en_GB}
\end{center}

\section{The Tabulation Scheme in this Document}

In order to assist in understanding how tabulation works, I explain here how I decided to produce the tables used in this document. They are deliberately not fully comprehensive, to show how one can select those tables that matter.

I started off by turning on indexing:

\begin{verbatim}
\usepackage[style=oscola,indexing=cite]{biblatex}
\end{verbatim}

I also loaded the package \textsc{imakeidx}, and since I intended to use \textsc{splitindex}, I did it with the appropriate option. Since I knew that my tables would be short, and I wanted to save space, I also used the option \verb|nonewpage|

\begin{verbatim}
\usepackage[splitindex,nonewpage]{imakeidx}
\end{verbatim}

I then decided what indexes I would create. I wanted reasonably comprehensive tables of cases -- a separate table for UK cases (including English, Scottish and Northern Irish cases), for EU cases (both alphabetical and by number), for international cases, and for other cases. I decided to send ECHR cases to the international cases index.

\begin{verbatim}
\makeindex[name=ukcases, intoc=true,
           title={Table of UK Cases}]
\makeindex[name=eucasesa, intoc=true,
           title={Table of EU Cases (Alphabetical)}]
\makeindex[name=eucasesn, intoc=true,
           title={Table of EU Cases (Numerical)}]
\makeindex[name=intcases, intoc=true,
           title={Table of International Cases}]
\makeindex[name=ocases, intoc=true,
           title={Table of Cases from Other Jurisdictions}]
\end{verbatim}

I decided to have just two tables of legislation: one for all UK legislation (primary and secondary, including court rules), and one for EU legislation.

\begin{verbatim}
\makeindex[name=ukleg, intoc=true,
           title={Table of UK Legislation}]
\makeindex[name=euleg, intoc=true,
           title={Table EU Legislation and Treaties}]
\end{verbatim}

I decided to have a table of treaties.
\begin{verbatim}
\makeindex[name=treaties, intoc=true,
           title={Table of Treaties}]
\end{verbatim}

And finally I decided to have a table of `Parliamentary Material', to which I would send draft legislation and all other UK Parliamentary material, such as Hansard.

\begin{verbatim}
\makeindex[name=pmats, intoc=true,
           title={Parliamentary Material and Draft Legislation}]
\end{verbatim}

With those indexes defined, I could not `hook them up'. All the UK cases go to the \texttt{ukcases} index.
\begin{verbatim}
\DeclareIndexAssociation{gbcases}{ukcases}% Supreme Court
\DeclareIndexAssociation{encases}{ukcases}% England
\DeclareIndexAssociation{sccases}{ukcases}% Scotland
\DeclareIndexAssociation{nicases}{ukcases}% Northern Ireland
\end{verbatim}

And the EU cases get sent to the alphabetically and numerically organised indexes as appropriate.
\begin{verbatim}
\DeclareIndexAssociation{eucases}{eucasesa}%    Alphabetical
\DeclareIndexAssociation{eucasesnum}{eucasesn}% Numeric
\end{verbatim}

ECHR and international cases are both sent to a single index, \texttt{intcases}
\begin{verbatim}
\DeclareIndexAssociation{echrcases}{intcases}%     ECHR
\DeclareIndexAssociation{echrcasescomm}{intcases}% ECHR Commission
\DeclareIndexAssociation{pilcases}{intcases}%      International
\end{verbatim}

And finally, all other cases (for instance US, Canadian, Australian) are sent to a fourth index.

\begin{verbatim}
\DeclareIndexAssociation{othercases}{ocases}
\end{verbatim}

Then we have to deal with legislation. UK (including English, Northern Irish, Scottish and Welsh) legislation, both primary and secondary, is sent to a single index, as are English rules of court.
\begin{verbatim}
\DeclareIndexAssociation{gbprimleg}{ukleg}
\DeclareIndexAssociation{gbsecleg}{ukleg}
\DeclareIndexAssociation{enprimleg}{ukleg}
\DeclareIndexAssociation{ensecleg}{ukleg}
\DeclareIndexAssociation{scprimleg}{ukleg}
\DeclareIndexAssociation{scsecleg}{ukleg}
\DeclareIndexAssociation{cyprimleg}{ukleg}
\DeclareIndexAssociation{cysecleg}{ukleg}
\DeclareIndexAssociation{niprimleg}{ukleg}
\DeclareIndexAssociation{nisecleg}{ukleg}
\DeclareIndexAssociation{enroc}{ukleg}
\end{verbatim}

EU legislation and treaties is all sent to another index.
\begin{verbatim}
\DeclareIndexAssociation{eutreaty}{euleg}
\DeclareIndexAssociation{euregs}{euleg}
\DeclareIndexAssociation{eudirs}{euleg}
\DeclareIndexAssociation{eudecs}{euleg}
\end{verbatim}

Treaties (other than the EU foundational treaties) go to yet another index.

\begin{verbatim}
\DeclareIndexAssociation{piltreaty}{treaties}
\end{verbatim}

And lastly, parliamentary materials and draft legislation go to the final index.
\begin{verbatim}
\DeclareIndexAssociation{gbdraftleg}{pmats}
\DeclareIndexAssociation{gbparltmat}{pmats}
\end{verbatim}

With that, all the indexes we want are hooked up. Everything else is going to go to the trash index, which is not going to be printed. I did however make one change to my bibliography database, in order to make sure that example \ref{spbill4} got indexed properly, I set its \texttt{tabulate} field to \texttt{pmats}, and gave it the option \texttt{skipbib}, so that it won't appear in the bibliography.

That being done, the document simply contains \verb|\cite| commands, which do their work automatically --- apart from the one example of a ship's name that I add at page \pageref{antaios85}.

At the end of the document, we specify the indexes to be printed.

\begin{verbatim}
\printindex[ukleg]    % UK Legislation
\printindex[ukcases]  % UK Cases
\printindex[euleg]    % EU Legislation and treaties
\printindex[eucasesa] % EU Cases, alphabetical
\printindex[eucasesn] % EU Cases, numerical
\printindex[treaties] % Treaties
\printindex[intcases] % International cases
\printindex[ocases]   % All other cases
\printindex[pmats]    % UK parliamentary material and
\end{verbatim}

I then run \texttt{latex}, \texttt{biber}, \texttt{latex}, \texttt{splitindex} and finally \texttt{latex} again. The document's `general' index is generated separately using individual indexing commands. In order to deal with the styling, I run \textsc{splitindex} with the options \texttt{-- -s oscola} (which generates indexes with dot leaders), and then separately run \textsc{makeindex} on \texttt{oscola-documentation-general} to produce the general index.

\clearpage
\addcontentsline{toc}{section}{Table of Examples}
{\def\indexname{Table of Examples}

\begin{theindex}

References are to examples, which are numbered in the margin.
\bigskip
\item act \ref{ucta}, \ref{nia1965}
\item article \ref{craig05}, \ref{young09}, \ref{griffith01}, \ref{waldron06}
\subitem casenote \ref{jameel04}
\subitem newspaper \ref{croft}
\item book \ref{endicott09}, \ref{jones09}, \ref{bar00}
  \subitem of `authority' \ref{stair}, \ref{colitt}, \ref{blackstone}
  \subitem chapter in collection, \ref{pila10}, \ref{cartwright09}
  \subitem collection \ref{horder00}, \ref{hart08}, \ref{castermans09}
  \subitem encyclopaedia, \ref{halsbury5}, \ref{cross}
  \subitem looseleaf \ref{cross}
  \subitem reference work \ref{halsbury5}, \ref{halsbury5:57}, \ref{friedrich68}
  \subitem reprint \ref{hobbes}
  \subitem translation \ref{birks87}, \ref{kotz98}
  \subitem volumes, multiple \ref{bar00}

\item case
 \subitem House of Lords \ref{corr08}, \ref{page96}, \ref{barrett01a}
 \subitem English
   \subsubitem English Reports \ref{henly28}
   \subsubitem in general \ref{corr08}, \ref{page96}, \ref{barrett01a}
   \subsubitem newspaper report \ref{powick93}
   \subsubitem unreported case \ref{stubbs90}, \ref{calvert02}
 \subitem Scottish \ref{hislop42}, \ref{adams03}, \ref{dodds03},
    \ref{crofters02}, \ref{davidson05}, \ref{smart06}
 \subitem Supreme Court \ref{davidson05}

\item{casenote} \ref{jameel04}

\item Civil Procedure Rules (CPR) \ref{cpr} \ref{pd}

\item COMDOC \ref{com13}, \ref{com348}

\item conference \ref{reliance}

\item County Court Rules (CCR) \ref{ccr}

\item email \ref{amazon:email}

\item electronic journal \ref{greenleaf10}, \ref{boyle04}

\item interview \ref{honore:interview}

\item legislation
\subitem draft \ref{confund}, \ref{confund2}, \ref{academies}, \ref{spbill4}, \ref{com895} 
\subitem explanatory notes \ref{charitiesnotes}
\subitem EU \ref{teu}, \ref{2002/60}
\subitem primary \ref{ucta}, \ref{nia1965}
\subsubitem welsh \ref{learner08}
\subitem secondary \ref{disorderly}, \ref{hollowware}, \ref{eggs}


\item letter \ref{brown:letter} \ref{amazon:email}

\item online material 
  \subitem book \ref{melbourne}
  \subitem electronic journal \ref{greenleaf10}, \ref{boyle04}
  \subitem newspaper article \ref{loader}
  \subitem website \ref{cole09}

\item practice direction \ref{pd}

\item procedure rules \ref{cpr}, \ref{rsc}, \ref{ccr}

\item report \ref{franks66}
\subitem command paper \ref{capital}, \ref{autumnperf}, \ref{eliminating:poverty}
\subitem Hansard \ref{hansardhc}, \ref{foia:2r}, \ref{healthbill}
\subitem Law Commission \ref{lawcom313}
\subitem parliamentary committees \ref{healthbill}, \ref{genomic}, \ref{equality}
\subitem Scottish Law Commission \ref{scotlawcom196}


\item rules of court \ref{cpr}, \ref{rsc}, \ref{ccr}

\item Rules of the Supreme Court \ref{rsc}

\item thesis \ref{herberg89}

\item treaty
  \subitem European Convention on Human Rights \ref{echr:treaty}
  \subitem European Union treaties \ref{teu}
  \subitem international treaties \ref{scheldt}, \ref{aaland}, \ref{mongolia}, \ref{echr:treaty}
  
\item UN documents
  \subitem by UN Doc number \ref{Indep}, \ref{res51_210}
  \subitem in ILM \ref{badinter4}

\item website \ref{cole09}
  

\end{theindex}
}



\clearpage

\printindex[ukleg]

\printindex[ukcases]

\printindex[euleg]

\printindex[eucasesa]

\printindex[eucasesn]

\printindex[treaties]

\clearpage

\printindex[intcases]

\printindex[ocases]

\clearpage

\printindex[pmats]

\clearpage

\addcontentsline{toc}{section}{Bibliography}

\printbibliography[nottype=jurisdiction,
                   nottype=legislation,
                   nottype=legal,
                   notsubtype=parliamentary,
                   nottype=commentary]

\clearpage

\printindex[general]

\end{document}
